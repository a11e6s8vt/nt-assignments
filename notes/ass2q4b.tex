%% Created by Maple 2023.2, Mac OS X
%% Source Worksheet: ass2q4b
%% Generated: Sun Jan 07 19:21:26 GMT 2024
\documentclass{article}
\usepackage{amssymb}
\usepackage{graphicx}
\usepackage{hyperref}
\usepackage{listings}
\usepackage{mathtools}
\usepackage{maple}
\usepackage[utf8]{inputenc}
\usepackage[svgnames]{xcolor}
\usepackage{amsmath}
\usepackage{breqn}
\usepackage{textcomp}
\begin{document}
\lstset{basicstyle=\ttfamily,breaklines=true,columns=flexible}
\pagestyle{empty}
\DefineParaStyle{Maple Bullet Item}
\DefineParaStyle{Maple Heading 1}
\DefineParaStyle{Maple Warning}
\DefineParaStyle{Maple Heading 4}
\DefineParaStyle{Maple Heading 2}
\DefineParaStyle{Maple Heading 3}
\DefineParaStyle{Maple Dash Item}
\DefineParaStyle{Maple Error}
\DefineParaStyle{Maple Title}
\DefineParaStyle{Maple Ordered List 1}
\DefineParaStyle{Maple Text Output}
\DefineParaStyle{Maple Ordered List 2}
\DefineParaStyle{Maple Ordered List 3}
\DefineParaStyle{Maple Normal}
\DefineParaStyle{Maple Ordered List 4}
\DefineParaStyle{Maple Ordered List 5}
\DefineCharStyle{Maple 2D Output}
\DefineCharStyle{Maple 2D Input}
\DefineCharStyle{Maple Maple Input}
\DefineCharStyle{Maple 2D Math}
\DefineCharStyle{Maple Hyperlink}
\begin{Maple Normal}
{$ \displaystyle \textit{find\_log} \coloneqq \boldsymbol{\mathrm{proc}}(n ,a ,m)
\\
 \boldsymbol{\mathrm{description}}\text{``Find log of a''};
\\
 \boldsymbol{\mathrm{for}}i \boldsymbol{\mathrm{from}}1\boldsymbol{\mathrm{to}}n 
\\
 \boldsymbol{\mathrm{do}}
\\
 \boldsymbol{\mathrm{if}}a \cdot i \boldsymbol{\mod}n =m 
\\
 \boldsymbol{\mathrm{then}}
\\
 \boldsymbol{\mathrm{return}}i ;
\\
 \boldsymbol{\mathrm{fi}};
\\
 \boldsymbol{\mathrm{end}}\boldsymbol{\mathrm{do}};
\\
 \boldsymbol{\mathrm{end}}\boldsymbol{\mathrm{proc}}; $}
\end{Maple Normal}

{$ \displaystyle  $}\begin{lstlisting}
> 
\end{lstlisting}
\begin{Maple Normal}

\end{Maple Normal}
\mapleinput
{$ \displaystyle \mathit{ifactor} (3267) $}

% \mapleresult
\begin{dmath}\label{(1)}
\left(3\right)^{3} \left(11\right)^{2}
\end{dmath}

{$ \displaystyle n \coloneqq 3^{3}\cdot 11^{2}; $}
{$ \displaystyle \, $}% \mapleresult
\begin{dmath}\label{(2)}
n \coloneqq 3267
\end{dmath}
\begin{Maple Normal}
g = x^13, h = x^157, n = 3267; <g> generatees the group of order 3267;


Steps 1:


\end{Maple Normal}
\mapleinput
{$ \displaystyle \mathit{g1} \coloneqq x^{13\cdot 11^{2}\mod \,27}; $}

% \mapleresult
\begin{dmath}\label{(3)}
\mathit{g1} \coloneqq x^{7}
\end{dmath}
\mapleinput
{$ \displaystyle \mathit{h1} \coloneqq x^{157\cdot 11^{2}\mod \,27}; $}

% \mapleresult
\begin{dmath}\label{(4)}
\mathit{h1} \coloneqq x^{16}
\end{dmath}
\begin{Maple Normal}
We need to find the log of h1 = x^16 in the cyclic group of order 27 generated by g1 = x^7. By trial and error, we get log(h1) = 10
\end{Maple Normal}
\begin{Maple Normal}

\end{Maple Normal}

{$ \displaystyle \textit{find\_log} (27,7,16); $}\begin{lstlisting}
> 
\end{lstlisting}
% \mapleresult
\begin{dmath}\label{(5)}
10
\end{dmath}
\begin{Maple Normal}
So our first congruence is x1 = 10 mod 27 --- (1)
\end{Maple Normal}
\begin{Maple Normal}

\end{Maple Normal}
\begin{Maple Normal}
Step 2:
\end{Maple Normal}
\begin{Maple Normal}

\end{Maple Normal}
\mapleinput
{$ \displaystyle \mathit{g2} \coloneqq x^{13\cdot 3^{3}\mod \,121}; $}

% \mapleresult
\begin{dmath}\label{(6)}
\mathit{g2} \coloneqq x^{109}
\end{dmath}
\mapleinput
{$ \displaystyle \mathit{h2} \coloneqq x^{157\cdot 3^{3}\mod \,121}; $}

% \mapleresult
\begin{dmath}\label{(7)}
\mathit{h2} \coloneqq x^{4}
\end{dmath}
\begin{Maple Normal}
We need to find the log of h2 = x^4 in the cyclic group of order 121 generated by g2 = x^109; using the proc find\_log above, it is = 118
\end{Maple Normal}
\mapleinput
{$ \displaystyle \textit{find\_log} (121,109,4); $}

% \mapleresult
\begin{dmath}\label{(8)}
40
\end{dmath}
\begin{Maple Normal}
We get our second congruence as: x2 = 40 mod 121 --- (2)
\end{Maple Normal}
\begin{Maple Normal}

\end{Maple Normal}
\begin{Maple Normal}
Hence  we need to find the unique solution to x = 10 mod 27, and x = 40 mod 121 using Chinese Remainder Theorem.
\end{Maple Normal}
\begin{Maple Normal}

\end{Maple Normal}
\mapleinput
{$ \displaystyle \mathit{with} (\mathit{NumberTheory}); $}

% \mapleresult
\begin{dmath}\label{(9)}
\left[\mathit{AreCoprime} ,\mathit{CalkinWilfSequence} ,\mathit{CarmichaelLambda} ,
\\
\mathit{ChineseRemainder} ,\mathit{ContinuedFraction} ,
\\
\mathit{ContinuedFractionPolynomial} ,\mathit{CyclotomicPolynomial} ,
\\
\mathit{Divisors} ,\mathit{FactorNormEuclidean} ,\mathit{HomogeneousDiophantine} ,
\\
\mathit{ImaginaryUnit} ,\mathit{InhomogeneousDiophantine} ,\mathit{IntegralBasis} ,
\\
\mathit{InverseTotient} ,\mathit{IsCyclotomicPolynomial} ,\mathit{IsMersenne} ,
\\
\mathit{IsSquareFree} ,\mathit{IthFermat} ,\mathit{IthMersenne} ,\mathit{JacobiSymbol} ,
\\
\mathit{JordanTotient} ,\mathit{KroneckerSymbol} ,\mathit{Landau} ,\mathit{LargestNthPower} ,
\\
\mathit{LegendreSymbol} ,M\ddot{o}\mathit{bius} ,\mathit{ModExtendedGCD} ,\mathit{ModularLog} ,
\\
\mathit{ModularRoot} ,\mathit{ModularSquareRoot} ,\mathit{Moebius} ,
\\
\mathit{MultiplicativeOrder} ,M\ddot{o}\mathit{bius} ,\mathit{NearestLatticePoint} ,
\\
\mathit{NextSafePrime} ,\mathit{NumberOfIrreduciblePolynomials} ,
\\
\mathit{NumberOfPrimeFactors} ,\Omega ,\Phi ,\mathit{PrimeCounting} ,\mathit{PrimeFactors} ,
\\
\mathit{PrimitiveRoot} ,\mathit{PseudoPrimitiveRoot} ,\mathit{QuadraticResidue} ,
\\
\mathit{Radical} ,\mathit{RepeatingDecimal} ,\mathit{RootsOfUnity} ,
\\
\mathit{SimplestRational} ,\mathit{SumOfDivisors} ,\mathit{SumOfSquares} ,\mathit{ThueSolve} ,
\\
\mathit{Totient} ,\lambda ,\mu ,\phi ,\mathrm{pi},\sigma ,\tau ,\varphi \right]
\end{dmath}
\begin{Maple Normal}

\end{Maple Normal}
\mapleinput
{$ \displaystyle \mathit{ChineseRemainder} ([10,40],[27,121]); $}

% \mapleresult
\begin{dmath}\label{(10)}
766
\end{dmath}
\begin{Maple Normal}

\end{Maple Normal}
\end{document}
