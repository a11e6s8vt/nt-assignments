\documentclass{article}
\title{General Guidelines on Browsing the Source Code}
\author{Ajeesh T Vijayan  \\
   MSc Cryptography  \\
   London Metropolitan University
}

\date{\today}
\begin{document}

    \maketitle

    \section*{Notes}
    This document describes how to browse the code developed as part of the
    Number Theory assignment.

    \begin{enumerate}
        \item Assignment answer sheets are named as:
            \begin{itemize}
                \item ajeesh\_msc\_crypto\_ass1.pdf,
                \item ajeesh\_msc\_crypto\_ass2.pdf,
                \item ajeesh\_msc\_crypto\_ass2.pdf
            \end{itemize}
        \item "src" directory contains the source code. "main.rs" is where the all the control flow starts
            \begin{itemize}
                \item primality.rs contains all primality related code,
                \item groups\_modulo.rs has all the primitive roots related code,
                \item quadratic\_sieve.rs contains the method to print the matrix for linear dependency calc.
                \item prime\_factors.rs contains code for prime factoring. It import methods from primality.rs
                \item logarithms.rs contains the code for Pollards Rho method
                \item presets.rs is a wrapper module for many of the methods in the above file. For example, when we need factorisation for a range of numbers, we write a wrapper in presets.rs which calls methods from primality.rs and prime\_factors.rs in loop
                \item utils.rs contains code for modular exponentiation, gcd calc, etc.
                \item cli\_ops defines the command line options
            \end{itemize}
        \item "notes" directory contains all the "latex" files.
        \item "results" directory has some sample json files from code execution.
        \item nt-assignments.exe is the executable file
    \end{enumerate}


\end{document}