%% Created by Maple 2023.2, Mac OS X
%% Source Worksheet: Pohlig_Hellman
%% Generated: Sun Jan 07 14:18:20 GMT 2024
\documentclass{article}
\usepackage{amssymb}
\usepackage{graphicx}
\usepackage{hyperref}
\usepackage{listings}
\usepackage{mathtools}
\usepackage{maple}
\usepackage[utf8]{inputenc}
\usepackage[svgnames]{xcolor}
\usepackage{amsmath}
\usepackage{breqn}
\usepackage{textcomp}
\begin{document}
\lstset{basicstyle=\ttfamily,breaklines=true,columns=flexible}
\pagestyle{empty}
\DefineParaStyle{Maple Bullet Item}
\DefineParaStyle{Maple Heading 1}
\DefineParaStyle{Maple Warning}
\DefineParaStyle{Maple Heading 4}
\DefineParaStyle{Maple Heading 2}
\DefineParaStyle{Maple Heading 3}
\DefineParaStyle{Maple Dash Item}
\DefineParaStyle{Maple Error}
\DefineParaStyle{Maple Title}
\DefineParaStyle{Maple Ordered List 1}
\DefineParaStyle{Maple Text Output}
\DefineParaStyle{Maple Ordered List 2}
\DefineParaStyle{Maple Ordered List 3}
\DefineParaStyle{Maple Normal}
\DefineParaStyle{Maple Ordered List 4}
\DefineParaStyle{Maple Ordered List 5}
\DefineCharStyle{Maple 2D Output}
\DefineCharStyle{Maple 2D Input}
\DefineCharStyle{Maple Maple Input}
\DefineCharStyle{Maple 2D Math}
\DefineCharStyle{Maple Hyperlink}
\mapleinput
{$ \displaystyle \texttt{>\,} n \coloneqq 3^{3}\cdot 5^{2}\cdot 23 $}

% \mapleresult
\begin{dmath}\label{(1)}
n \coloneqq 15525
\end{dmath}
\begin{Maple Normal}
We choose g = x^77 as the generator of G the cyclic group of order 15525 (we know g generates G as 77 has no common factors with 15525)
\end{Maple Normal}
\mapleinput
{$ \displaystyle \texttt{>\,} g \coloneqq x^{77}; $}

% \mapleresult
\begin{dmath}\label{(2)}
g \coloneqq x^{77}
\end{dmath}
\begin{Maple Normal}
We aim to find the logarithm of h = x^372 in G where G is generated by g
\end{Maple Normal}
\begin{Maple Normal}

\end{Maple Normal}
\begin{Maple Normal}
Step 1
\end{Maple Normal}
\mapleinput
{$ \displaystyle \texttt{>\,} \mathit{g1} \coloneqq x^{77\cdot 5^{2}\cdot 23\,\mod \,27} $}

% \mapleresult
\begin{dmath}\label{(3)}
\mathit{g1} \coloneqq x^{22}
\end{dmath}
\mapleinput
{$ \displaystyle \texttt{>\,} \mathit{h1} \coloneqq x^{372\cdot 5^{2}\cdot 23\,\mod \,27} $}

% \mapleresult
\begin{dmath}\label{(4)}
\mathit{h1} \coloneqq x^{6}
\end{dmath}
\begin{Maple Normal}
We need to find the log of h1 =  x^6 in the cyclic group of order 27 generated by g1 = x^22. Using the basic Pohling Hellman algorithm for prime powers or otherwise we find that this is 15 which we can see is true as
\end{Maple Normal}
\begin{Maple Normal}

\end{Maple Normal}
\mapleinput
{$ \displaystyle \texttt{>\,} x^{22\cdot 15\,\mod \,27} $}

% \mapleresult
\begin{dmath}\label{(5)}
x^{6}
\end{dmath}
\mapleinput
{$ \displaystyle \texttt{>\,} \, $}

\begin{Maple Normal}
So our first congruence is x1 = 15 mod 27
\end{Maple Normal}
\begin{Maple Normal}

\end{Maple Normal}
\begin{Maple Normal}
Step 2
\end{Maple Normal}
\mapleinput
{$ \displaystyle \texttt{>\,} \mathit{g2} \coloneqq x^{77\cdot 3^{3}\cdot 23\,\mod \,25} $}

% \mapleresult
\begin{dmath}\label{(6)}
\mathit{g2} \coloneqq x^{17}
\end{dmath}
\mapleinput
{$ \displaystyle \texttt{>\,} \,\mathit{h2} \coloneqq x^{372\cdot 3^{3}\cdot 23\,\mod \,25} $}

% \mapleresult
\begin{dmath}\label{(7)}
\mathit{h2} \coloneqq x^{12}
\end{dmath}
\begin{Maple Normal}
We need to find the log of h2 =  x^12 in the cyclic group of order 25 generated by g2 = x^17 ; we see that this is 11 as
\end{Maple Normal}
\mapleinput
{$ \displaystyle \texttt{>\,} x^{17\cdot 11\,\mod \,25} $}

% \mapleresult
\begin{dmath}\label{(8)}
x^{12}
\end{dmath}
\mapleinput
{$ \displaystyle \texttt{>\,} \, $}

\begin{Maple Normal}
Hence the log of h2 mod 25 is 11 giving x2 = 11 mod 25 as our second congruence
\end{Maple Normal}
\begin{Maple Normal}

\end{Maple Normal}
\begin{Maple Normal}
Step 3
\end{Maple Normal}
\begin{Maple Normal}

\end{Maple Normal}
\mapleinput
{$ \displaystyle \texttt{>\,} \, $}

\mapleinput
{$ \displaystyle \texttt{>\,} \mathit{g3} \coloneqq x^{77\cdot 3^{3}\cdot 5^{2}\,\mod \,23} $}

% \mapleresult
\begin{dmath}\label{(9)}
\mathit{g3} \coloneqq x^{18}
\end{dmath}
\mapleinput
{$ \displaystyle \texttt{>\,} \,\mathit{h3} \coloneqq x^{372\cdot 3^{3}\cdot 5^{2}\,\mod \,23} $}

% \mapleresult
\begin{dmath}\label{(10)}
\mathit{h3} \coloneqq x^{9}
\end{dmath}
\begin{Maple Normal}
We need to find the log of h3=  x^9 in the cyclic group of order 23 generated by g2 = x^18; we see that this is 12 as
\end{Maple Normal}
\mapleinput
{$ \displaystyle \texttt{>\,} x^{18\cdot 12\,\mod \,23} $}

% \mapleresult
\begin{dmath}\label{(11)}
x^{9}
\end{dmath}
\mapleinput
{$ \displaystyle \texttt{>\,}  $}

\begin{Maple Normal}
Hence the log of h3 mod 23 is 12 giving x3 = 12 mod 23 as our 3rd congruence
\end{Maple Normal}
\begin{Maple Normal}

\end{Maple Normal}
\begin{Maple Normal}
Hence we need to find the unique solution to x = 15 mod 27, x = 11 mod 25, x = 12 mod 23. By the Chinese Remainder Theorem or otherwise we find x = 10086 mod 15525.
\end{Maple Normal}
\begin{Maple Normal}

\end{Maple Normal}
\begin{Maple Normal}
We claim that the log of x^372 in the cyclic group of order 15525 with generator g = x^77 is x = 10086. We can see that this is true as 
\end{Maple Normal}
\begin{Maple Normal}

\end{Maple Normal}
\begin{Maple Normal}

\end{Maple Normal}
\mapleinput
{$ \displaystyle \texttt{>\,} 77\cdot 10086\,\mod \,15525 $}

% \mapleresult
\begin{dmath}\label{(12)}
372
\end{dmath}
\mapleinput
{$ \displaystyle \texttt{>\,}  $}

\mapleinput
{$ \displaystyle \texttt{>\,}  $}

\end{document}
