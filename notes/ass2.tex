% Assignment1 - Ajeesh T. Vijayan
\documentclass[11pt,a4paper]{article}
\usepackage[utf8]{inputenc}
\usepackage{array}
\usepackage{caption}
\usepackage{enumerate}
\usepackage{amsmath, amssymb}
\usepackage{array, makecell}
\usepackage{booktabs}
\usepackage{graphicx}
\usepackage[left=2.5cm,right=1.5cm,top=2cm,bottom=1.5cm]{geometry}
\usepackage{wrapfig}
\usepackage{float}
\usepackage{fancyhdr}
\usepackage[colorlinks=true]{hyperref}
\usepackage{listings, lstautogobble}
\usepackage{listings-rust}
\usepackage[verbatim]{lstfiracode}
\usepackage{xcolor}
\usepackage{fontawesome5}
\usepackage{adjustbox}

%New colors defined below
\definecolor{codegreen}{rgb}{0,0.6,0}
\definecolor{codegray}{rgb}{0.5,0.5,0.5}
\definecolor{codepurple}{rgb}{0.58,0,0.82}
\definecolor{backcolour}{rgb}{0.95,0.95,0.92}

%Code listing style named "mystyle"
\lstdefinestyle{mystyle}{
	mathescape=true,
	language=Rust,
	backgroundcolor=\color{backcolour},
	style=FiraCodeStyle,   
	commentstyle=\color{codegreen},
	keywordstyle=\color{magenta},
	numberstyle=\tiny\color{codegray},
	stringstyle=\color{codepurple},
	basicstyle=\ttfamily\small,
	breakatwhitespace=false,         
	breaklines=true,                 
	captionpos=b,                    
	keepspaces=true,                 
	numbers=left,                    
	numbersep=3pt,                  
	showspaces=false,                
	showstringspaces=false,
	showtabs=false,                  
	tabsize=2
}

%"mystyle" code listing set
\lstset{style=mystyle}

\lstdefinestyle{DOS}
{
	backgroundcolor=\color{black},
	basicstyle=\scriptsize\color{white}\ttfamily
}

\colorlet{punct}{red!60!black}
\definecolor{background}{HTML}{EEEEEE}
\definecolor{delim}{RGB}{20,105,176}
\colorlet{numb}{magenta!60!black}

\lstdefinelanguage{json}{
	mathescape=true,
	basicstyle=\normalfont\ttfamily,
	numbers=left,
	numberstyle=\scriptsize,
	stepnumber=1,
	numbersep=8pt,
	showstringspaces=false,
	breaklines=true,
	frame=lines,
	backgroundcolor=\color{background},
	literate=
	*{0}{{{\color{numb}0}}}{1}
	{1}{{{\color{numb}1}}}{1}
	{2}{{{\color{numb}2}}}{1}
	{3}{{{\color{numb}3}}}{1}
	{4}{{{\color{numb}4}}}{1}
	{5}{{{\color{numb}5}}}{1}
	{6}{{{\color{numb}6}}}{1}
	{7}{{{\color{numb}7}}}{1}
	{8}{{{\color{numb}8}}}{1}
	{9}{{{\color{numb}9}}}{1}
	{:}{{{\color{punct}{:}}}}{1}
	{,}{{{\color{punct}{,}}}}{1}
	{\{}{{{\color{delim}{\{}}}}{1}
	{\}}{{{\color{delim}{\}}}}}{1}
	{[}{{{\color{delim}{[}}}}{1}
	{]}{{{\color{delim}{]}}}}{1},
}

\makeatletter
\newcommand{\github}[1]{%
	\href{#1}{\faGithubSquare}%
}
\makeatother

\captionsetup[table]{position=bottom}   %% or below

\pagestyle{fancy}
\lhead{Ajeesh T. Vijayan}
\rhead{Student No: 22077273}
\cfoot{\thepage}
\renewcommand{\headrulewidth}{0.4pt}
\renewcommand{\footrulewidth}{0.4pt}

\title{MA7010 – Number Theory for Cryptography - Assignment 2}
\author{Ajeesh Thattukunnel Vijayan}
\date{January 11\textsuperscript{th} 2024}

\definecolor{dkgreen}{rgb}{0,0.6,0}
\definecolor{gray}{rgb}{0.5,0.5,0.5}
\definecolor{mauve}{rgb}{0.58,0,0.82}

\mathchardef\mathcomma=\mathcode`,
\newcommand{\roverline}[1]{\mathpalette\doroverline{#1}}
\newcommand{\doroverline}[2]{\overline{#1#2}}

\begin{document}
	
	\maketitle
	\section{Answers} \label{sec:Forces}
	
	\begin{enumerate}[1.]
		\item $\text{Lower Range} = 600, \text{Upper Range} = 750$. Consider all the numbers n in your range. Divide the set into two subsets:
		A - the subset consisting of all n where there is at least one primitive root modulo n; B – the subset consisting of all n where no primitive roots exist modulo n
		\bigskip
		
		\item 
		\begin{enumerate}[a.]
			\item Explain why we can always find a primitive root modulo p when p is a prime.
			\item Express the number of primitive roots that exist modulo p using the Euler Totient function and show that your answer correctly predicts the number of primitive roots for all primes in your given range.
			\item For the same range as Question 1 use the command ifactors in Maple to find the set C whose elements consist of numbers of the form $p^k(p > 2, k \ge 1)$ or $2p^k(p > 2, k \ge 1)$
			\item Hence form a conjecture about when primitive roots do and don’t exist
		\end{enumerate}
			

		\item Suppose n has the form n = pq where p and q are different primes both > 2. \\
	
		\begin{enumerate}[(a)]
			\item What is $\phi(n)$ in terms of $p$ and $q$?
			\item Suppose $a$ is relatively prime to $pq$.  Explain why 
			\begin{enumerate}[i.]
				\item $a^{p-1} \equiv 1\mod p$
				\item $a^{q-1} \equiv 1\mod q$
				\item $m = lcm(p-1, q-1)$ is less than $(p-1)(q-1)$
				\item $a^m \equiv 1 \mod (p-1)(q-1)$
			\end{enumerate}
			\item Hence explain why numbers of the form n have no primitive roots. \href{https://math.stackexchange.com/questions/162157/for-q-p-odd-primes-such-that-p-neq-q-there-is-not-primitive-root-modulo}{check it out}
			\item Show that all numbers of the form $n = pq$ (p and q both odd primes) in your range are included in set B.
		\end{enumerate}
		\item Use the BabyStepsGiantSteps algorithm to find discrete logarithms x of b mod n for the primitive root a for each of the two examples assigned to you in the table below. Verify that your answer is correct by calculating $a^x \mod m$ by hand using the method of modular exponentiation.
		\item Use the Pohlig Helmann algorithm to find in the cyclic group of order n with the generating element a for both the examples assigned to you below. Verify your answer in Maple.
		\item Use the Pollard Rho method to verify your answer to the first example you were allocated in Question 4. 
		
		\begin{table}[H]
			\begin{adjustbox}{scale=0.9,center}
				\begin{tabular}{ |p{2cm}|p{2cm}|p{2cm}|p{2cm}|p{4cm}| }
					\hline
					Name & b & n & a & Method \\
					\hline
					Ajeesh & $47$   & 71   & $21$   & BabyStepGiantStep \\
					Ajeesh & $24$   & 53   & $26$   & BabyStepGiantStep  \\
					Ajeesh & $x^{41}$ & 343  & $x^{11}$ & Pohlig Hellmen \\
					Ajeesh & $x^{157}$& 3267 & $x^{13}$ & Pohlig Hellmen  \\
					\hline
				\end{tabular}
			\end{adjustbox}
			\caption{List of composite numbers of the form P.Q}
			\label{table:composite-pq}
		\end{table}
		
	\end{enumerate}
	
	\begin{thebibliography}{unsrt}
		
		\bibitem{Modular_Mathematics}
		C R Jordan \& D A Jordan \emph{MODULAR MATHEMATICS Groups }.
		
		\bibitem{gt_solutions}
		Dr. Ben Fairbairn \emph{GROUP THEORY Solutions to Exercises}.
		
		\bibitem{online_ref_1}
		\emph{https://github.com/Ssophoclis/AKS-algorithm/blob/master/AKS.py}
		
	\end{thebibliography}
	
\end{document}