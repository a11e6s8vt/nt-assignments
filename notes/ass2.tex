% Assignment1 - Ajeesh T. Vijayan
\documentclass[11pt,a4paper,fleqn]{article}
\usepackage[utf8]{inputenc}
\usepackage{array}
\usepackage{caption}
\usepackage{enumerate}
\usepackage{amsmath, amssymb}
\usepackage{array, makecell}
\usepackage{booktabs}
\usepackage{graphicx}
\usepackage[left=2.5cm,right=1.5cm,top=2cm,bottom=1.5cm]{geometry}
\usepackage{wrapfig}
\usepackage{float}
\usepackage{fancyhdr}
\usepackage[colorlinks=true]{hyperref}
\usepackage{listings, lstautogobble}
\usepackage{listings-rust}
\usepackage[verbatim]{lstfiracode}
\usepackage{xcolor}
\usepackage{fontawesome5}
\usepackage{adjustbox}
\usepackage{tabularx, makecell, booktabs}%
\usepackage{maple}
\usepackage{breqn}
\usepackage{textcomp}
\usepackage{graphicx}
\usepackage{hyperref}
\usepackage{listings}
\usepackage{mathtools}
\usepackage{maple}
\usepackage[utf8]{inputenc}
\usepackage[svgnames]{xcolor}
\usepackage{breqn}
\usepackage{verbatimbox,caption,float,lipsum}
\usepackage{mdframed}
\newfloat{Code}
\captionsetup{Code}

%New colors defined below
\definecolor{codegreen}{rgb}{0,0.6,0}
\definecolor{codegray}{rgb}{0.5,0.5,0.5}
\definecolor{codepurple}{rgb}{0.58,0,0.82}
\definecolor{backcolour}{rgb}{0.95,0.95,0.92}

%Code listing style named "mystyle"
\lstdefinestyle{mystyle}{
	mathescape=true,
	language=Rust,
	backgroundcolor=\color{backcolour},
	style=FiraCodeStyle,   
	commentstyle=\color{codegreen},
	keywordstyle=\color{magenta},
	numberstyle=\tiny\color{codegray},
	stringstyle=\color{codepurple},
	basicstyle=\ttfamily\small,
	breakatwhitespace=false,         
	breaklines=true,                 
	captionpos=b,                    
	keepspaces=true,                 
	numbers=left,                    
	numbersep=3pt,                  
	showspaces=false,                
	showstringspaces=false,
	showtabs=false,                  
	tabsize=2
}

%"mystyle" code listing set
\lstset{style=mystyle}

\lstdefinestyle{DOS}
{
	backgroundcolor=\color{black},
	basicstyle=\scriptsize\color{white}\ttfamily
}

\colorlet{punct}{red!60!black}
\definecolor{background}{HTML}{EEEEEE}
\definecolor{delim}{RGB}{20,105,176}
\colorlet{numb}{magenta!60!black}

\lstdefinelanguage{json}{
	mathescape=true,
	basicstyle=\normalfont\ttfamily,
	numbers=left,
	numberstyle=\scriptsize,
	stepnumber=1,
	numbersep=8pt,
	showstringspaces=false,
	breaklines=true,
	frame=lines,
	backgroundcolor=\color{background},
	literate=
	*{0}{{{\color{numb}0}}}{1}
	{1}{{{\color{numb}1}}}{1}
	{2}{{{\color{numb}2}}}{1}
	{3}{{{\color{numb}3}}}{1}
	{4}{{{\color{numb}4}}}{1}
	{5}{{{\color{numb}5}}}{1}
	{6}{{{\color{numb}6}}}{1}
	{7}{{{\color{numb}7}}}{1}
	{8}{{{\color{numb}8}}}{1}
	{9}{{{\color{numb}9}}}{1}
	{:}{{{\color{punct}{:}}}}{1}
	{,}{{{\color{punct}{,}}}}{1}
	{\{}{{{\color{delim}{\{}}}}{1}
	{\}}{{{\color{delim}{\}}}}}{1}
	{[}{{{\color{delim}{[}}}}{1}
	{]}{{{\color{delim}{]}}}}{1},
}

\lstnewenvironment{myverbatim}[1][]{%
	\lstset{
		basicstyle=\ttfamily,
		frame=tb,
		#1
	}%
}{}

\makeatletter
\newcommand{\github}[1]{%
	\href{#1}{\faGithubSquare}%
}
\makeatother

\makeatletter
\newcommand{\tpmod}[1]{{\@displayfalse\pmod{#1}}}
\makeatother

\newtheorem{theorem}{Theorem}

% Define typographic struts, as suggested by Claudio Beccari
%   in an article in TeX and TUG News, Vol. 2, 1993.
\newcommand\Tstrut{\rule{0pt}{2.6ex}}         % = `top' strut
\newcommand\Bstrut{\rule[-0.9ex]{0pt}{0pt}}   % = `bottom' strut

\captionsetup[table]{position=bottom}   %% or below

\pagestyle{fancy}
\lhead{Ajeesh T. Vijayan}
\rhead{Student No: 22077273}
\cfoot{\thepage}
\renewcommand{\headrulewidth}{0.4pt}
\renewcommand{\footrulewidth}{0.4pt}

\title{MA7010 – Number Theory for Cryptography - Assignment 2}
\author{Ajeesh Thattukunnel Vijayan}
\date{January 11\textsuperscript{th} 2024}

\definecolor{dkgreen}{rgb}{0,0.6,0}
\definecolor{gray}{rgb}{0.5,0.5,0.5}
\definecolor{mauve}{rgb}{0.58,0,0.82}

\newcommand{\roverline}[1]{\mathpalette\doroverline{#1}}
\newcommand{\doroverline}[2]{\overline{#1#2}}

\begin{document}
	\DefineParaStyle{Maple Bullet Item}
	\DefineParaStyle{Maple Heading 1}
	\DefineParaStyle{Maple Warning}
	\DefineParaStyle{Maple Heading 4}
	\DefineParaStyle{Maple Heading 2}
	\DefineParaStyle{Maple Heading 3}
	\DefineParaStyle{Maple Dash Item}
	\DefineParaStyle{Maple Error}
	\DefineParaStyle{Maple Title}
	\DefineParaStyle{Maple Ordered List 1}
	\DefineParaStyle{Maple Text Output}
	\DefineParaStyle{Maple Ordered List 2}
	\DefineParaStyle{Maple Ordered List 3}
	\DefineParaStyle{Maple Normal}
	\DefineParaStyle{Maple Ordered List 4}
	\DefineParaStyle{Maple Ordered List 5}
	\DefineCharStyle{Maple 2D Output}
	\DefineCharStyle{Maple 2D Input}
	\DefineCharStyle{Maple Maple Input}
	\DefineCharStyle{Maple 2D Math}
	\DefineCharStyle{Maple Hyperlink}
	\maketitle
	\section{Answers} \label{sec:Forces}
	
	\begin{enumerate}[1.]
		\item $\text{Lower Range} = 600, \text{Upper Range} = 750$. Consider all the numbers n in your range. Divide the set into two subsets:
		A - the subset consisting of all n where there is at least one primitive root modulo n; B – the subset consisting of all n where no primitive roots exist modulo n
		\bigskip
		\begin{flushleft}
			\textbf{\textit{Answer:}} The below json snippets show the two sets:
			\begin{lstlisting}[
				language=json,
				firstnumber=1,
				caption={List of Numbers With and Without Primitive Roots},
				label={lst:nums-primitive-roots}
				]
				{
					"Numbers With Primitve Roots (A)": [ 
						"601","607","613","614","617","619","622","625",
						"626","631","634","641","643","647","653","659",
						"661","662","673","674","677","683","686","691",
						"694","698","701","706","709","718","719","722",
						"727","729","733","734","739","743","746"
					]
				}

				{
					"Numbers Without Primitive Roots (B)": [
						"600","602","603","604","605","606","608","609",
						"610","611","612","615","616","618","620","621",
						"623","624","627","628","629","630","632","633",
						"635","636","637","638","639","640","642","644",
						"645","646","648","649","650","651","652","654",
						"655","656","657","658","660","663","664","665",
						"666","667","668","669","670","671","672","675",
						"676","678","679","680","681","682","684","685",
						"687","688","689","690","692","693","695","696",
						"697","699","700","702","703","704","705","707",
						"708","710","711","712","713","714","715","716",
						"717","720","721","723","724","725","726","728",
						"730","731","732","735","736","737","738","740",
						"741","742","744","745","747","748","749","750"
					]
				}
			\end{lstlisting}
			
			The Rust code for generating the above result is below:
			\begin{lstlisting}[
				style={mystyle},
				caption={Primitve Roots Calculation},
				label={lst:primitive-roots}
				]
				
				///
				/// Returns a vec of primitive roots for the integer
				///
				/// # Arguments
				/// * n: BigInt
				///
				/// Steps:
				/// This function uses trial and error to find primitive roots 
				/// associated to an Integer
				///
				/// 1. Find all coprime numbers less than `n` 
				///    (coprime_nums_less_than_n)
				/// 2. $\phi(n)$ = total number of coprimes
				/// 3. Find all the divisors of $\phi(n)$. Order of an 
				///    element in the Modulo n group will be equal to 
				///    any of the divisor values.
				/// 4. Find the order of each of the coprimes to n one by one 
				///    (skip 1 from the list of
				///    coprimes as 1 is a trivial root) (`use utils::modular_pow)
				/// 5. if order of a coprime integer equals $\phi(n)$, that coprime 
				///    is a primitive root
				///
				/// The above steps are executed aginst all coprimes to n and 
				/// returns an integer vector with primitive roots
				///
				pub fn primitive_roots_trial_n_error(n: &BigInt) -> Vec<BigInt> {
					let mut primitive_roots: Vec<BigInt> = Vec::new();
					let mut has_primitive_roots: bool = false;
					
					let nums_coprime_n: Vec<BigInt> = coprime_nums_less_than_n(n);
					let phi_n = BigInt::from(nums_coprime_n.len());
					//
					let divisors_phi_n = divisors_of_n(&phi_n);
					
					for a in nums_coprime_n {
						let mut has_order_phi: bool = true;
						for order in divisors_phi_n.iter() {
							if modular_pow(&a, order, n) == BigInt::one() {
								if *order != phi_n {
									has_order_phi = false;
								}
							}
						}
						
						if has_order_phi {
							primitive_roots.push(a);
							has_primitive_roots = true;
							break;
						}
					}
					
					if has_primitive_roots {
						let orders_coprime_phi_n: Vec<BigInt> = coprime_nums_less_than_n(&phi_n);
						// first coprime number is 1 and we are skipping that 
						// when calculating power
						for order in orders_coprime_phi_n.iter().skip(1) {
							primitive_roots.push(modular_pow(&primitive_roots[0], order, n));
						}
					}
					
					primitive_roots.sort();
					
					for (i, num) in primitive_roots.clone().iter().enumerate() {
						if num == &BigInt::one() {
							primitive_roots.remove(i);
							continue;
						}
						
						if modular_pow(num, &phi_n, n) != BigInt::one() {
							primitive_roots.remove(i);
						}
					}
					
					primitive_roots
				}
				
				///
				/// Generates a list of integers less than n and co-prime to n.
				///
				pub fn coprime_nums_less_than_n(n: &BigInt) -> Vec<BigInt> {
					let mut coprimes: Vec<BigInt> = Vec::new();
					let r = range(BigInt::from(1u64), n.clone());
					
					for num in r {
						if n.gcd_euclid(&num) == BigInt::one() {
							coprimes.push(num)
						}
					}
					coprimes.sort();
					coprimes
				}
				
				///
				/// Get list of divisors of a number n > 2
				///
				pub fn divisors_of_n(n: &BigInt) -> Vec<BigInt> {
					let mut divisors: Vec<BigInt> = Vec::new();
					let mut primes = vec![BigInt::from(2u64)];
					let p_factors_n = n.prime_factors(&mut primes);
					let p_factors_n = p_factors_n
					.iter()
					.map(|(p, _)| p.clone())
					.collect::<Vec<BigInt>>();
					
					for p in p_factors_n {
						let mut i = 0;
						loop {
							let pow = p.pow(i);
							if n % &pow == BigInt::zero() {
								divisors.push(n / &pow);
								divisors.push(pow);
								i += 1;
							} else {
								break;
							}
						}
					}
					divisors.sort();
					divisors.dedup();
					divisors
				}
			\end{lstlisting}
			
		\end{flushleft}
		\item 
		\begin{enumerate}[a.]
			\item Explain why we can always find a primitive root modulo p when p is a prime.
			\bigskip
			\begin{flushleft}
				\textbf{\textit{Answer:}} (Not complete)
				\begin{theorem}[Euler's Theorem]
					\label{eulers_thm}
					Suppose that $m \ge 1$ and $(a, m) = 1$, then $a^{\phi(m)} = 1 (\mod m$), where $\phi(m)$ is Euler's Totient function which yields the number of integers less than $m$ and relatively prime to m.
				\end{theorem}
				
				A special case occurs when $m$ is a prime number, which is called Fermat's Little theorem. When $m$ is a prime, the number of integers less than $m$ and relatively prime to $m$ equal $m-1$. i.e., $\phi(m) = m - 1$.  
			\end{flushleft}
			\item Express the number of primitive roots that exist modulo p using the Euler Totient function and show that your answer correctly predicts the number of primitive roots for all primes in your given range.
			\begin{flushleft}
				\textbf{\textit{Answer:}} The number of primitive roots associated with an integer $n$ is given by $\phi(\phi(n))$
				When $n$ is a prime, namely $p$, $\phi(\phi(p)) = \phi(p-1)$. The below table verifies this value against the number calculated using trial and error for all primes in the range $600 \le p \le 750$.
				
				\begin{table}[!ht]
					\centering
					\begin{tabular}{|l|l|l|}
						\hline
						Prime & Primitive Roots Count - Trial and Error & $\phi(p-1)$ \\ \hline
						601 & 160 & 160 \\ \hline
						607 & 200 & 200 \\ \hline
						613 & 192 & 192 \\ \hline
						617 & 240 & 240 \\ \hline
						619 & 204 & 204 \\ \hline
						631 & 144 & 144 \\ \hline
						641 & 256 & 256 \\ \hline
						643 & 212 & 212 \\ \hline
						647 & 288 & 288 \\ \hline
						653 & 324 & 324 \\ \hline
						659 & 276 & 276 \\ \hline
						661 & 160 & 160 \\ \hline
						673 & 192 & 192 \\ \hline
						677 & 312 & 312 \\ \hline
						683 & 300 & 300 \\ \hline
						691 & 176 & 176 \\ \hline
						701 & 240 & 240 \\ \hline
						709 & 232 & 232 \\ \hline
						719 & 358 & 358 \\ \hline
						727 & 220 & 220 \\ \hline
						733 & 240 & 240 \\ \hline
						739 & 240 & 240 \\ \hline
						743 & 312 & 312 \\ \hline
					\end{tabular}
					\caption{Primitive Roots Count}
					\label{table:primitive_roots_count}
				\end{table}
				
				The Rust code for generating the above result is below. It calls the function listed in code~\ref{lst:primitive-roots}.
				\begin{lstlisting}[
					style={mystyle},
					caption={Primitive Roots - Euler's Totient Function Verification},
					label={lst:primitive-roots-totient-verification}
					]
					
					            PrimitiveRootsCommands::Ass2Question2b(r) => {
						let start = r.start;
						let end = r.end;
						
						let mut result: Vec<HashMap<String, String>> = Vec::new();
						let (primes_in_range, _) = find_primes_in_range_trial_division_parallel(start, end);
						for p in primes_in_range.iter() {
							let primitive_roots = primitive_roots_trial_n_error(p);
							let phi_phi_n = euler_totient_phi(&(p - BigInt::one()));
							let mut item: HashMap<String, String> = HashMap::new();
							item.insert("Prime".to_string(), p.to_string());
							item.insert("Euler_Totient(p-1)".to_string(), phi_phi_n.to_string());
							item.insert(
							"Prim Roots Count - Trial and Error".to_string(),
							primitive_roots.len().to_string(),
							);
							result.push(item);
						}
						println!("{}", serde_json::to_string_pretty(&result).unwrap())
					}
				\end{lstlisting}
				
				\bigbreak
				We can use the below command to see the above result:
				\begin{lstlisting}[style=DOS, caption=Verify Primitive Roots Counting using Totient Function]
					
				.\target\release\nt-assignments.exe primitive-roots ass2-question2b -s 600 -e 750
				\end{lstlisting}
			\end{flushleft}
			
			\item For the same range as Question 1 use the command ifactors in Maple to find the set C whose elements consist of numbers of the form $p^k(p > 2, k \ge 1)$ or $2p^k(p > 2, k \ge 1)$
			\begin{table}[H]
				\centering
				\begin{tabular}{|l|l|l|l|}
					\hline
					Number & Form & Number & Form\\ \hline
					601 & $601^1$ & 674 & $2^1 \times 337^1$  \\ \hline
					607 & $607^1$ & 677 & $677^1$  \\ \hline
					613 & $613^1$ & 683 & $683^1$  \\ \hline
					614 & $2^1 \times 307^1$ & 686 & $2^1 \times 7^3$  \\ \hline
					617 & $617^1$ & 691 & $691^1$  \\ \hline
					619 & $619^1$ & 694 & $2^1 \times 347^1$  \\ \hline
					622 & $2^1 \times 311^1$ & 698 & $2^1 \times 349^1$  \\ \hline
					625 & $5^4$ & 701 & $701^1$  \\ \hline
					626 & $2^1 \times 313^1$ & 706 & $2^1 \times 353^1$  \\ \hline
					631 & $631^1$ & 709 & $709^1$  \\ \hline
					634 & $2^1 \times 317^1$ & 718 & $2^1 \times 359^1$  \\ \hline
					641 & $641^1$ & 719 & $719^1$  \\ \hline
					643 & $643^1$ & 722 & $2^1 \times 19^2$  \\ \hline
					647 & $647^1$ & 727 & $727^1$  \\ \hline
					653 & $653^1$ & 729 & $3^6$  \\ \hline
					659 & $659^1$ & 733 & $733^1$  \\ \hline
					661 & $661^1$ & 734 & $2^1 \times 367^1$  \\ \hline
					662 & $2^1 \times 331^1$ & 739 & $739^1$  \\ \hline
					673 & $673^1$ & 743 & $743^1$  \\ \hline
					746 & $2^1 \times 373^1$ & - & - \\ \hline					
				\end{tabular}
				\caption{Numbers of the form $p^k$, $2p^k$}
				\label{table:number_pk_2pk}
			\end{table}

			\item Hence form a conjecture about when primitive roots do and don’t exist
		\end{enumerate}
			

		\item Suppose n has the form $n = pq$ where $p$ and $q$ are different primes both $> 2$. \\
	
		\begin{enumerate}[(a)]
			\item What is $\phi(n)$ in terms of $p$ and $q$?
			\begin{flushleft}
				\textbf{\textit{Answer:}} $\phi(n) = \phi(p.q) = \phi(p).\phi(q) = (p-1).(q-1)$
			\end{flushleft}
			\item Suppose $a$ is relatively prime to $pq$.  Explain why 
			\begin{enumerate}[i.]
				\item $a^{p-1} \equiv 1\mod p$
				\begin{flushleft}
					\textbf{\textit{Answer:}} Given $p$ and $q$ are distinct primes. Since $(a, pq) = 1$, $a$ is relatively prime to both $p$ and $q$. Hence by Fermat's Little Theorem, $a^{p-1} \equiv 1 mod p$.
				\end{flushleft}
				\item $a^{q-1} \equiv 1\mod q$
				\begin{flushleft}
					\textbf{\textit{Answer:}} Given $p$ and $q$ are distinct primes. Since $(a, pq) = 1$, $a$ is relatively prime to both $p$ and $q$. Hence by Fermat's Little Theorem, $a^{q-1} \equiv 1 mod q$.
				\end{flushleft}
				\item \label{item:Q3biii} $m = lcm(p-1, q-1)$ is less than $(p-1)(q-1)$ 
				\begin{flushleft}
					\textbf{\textit{Answer:}} Since both $p$ and $q$ are odd primes, $p-1$ and $q-1$ are even. Let $p - 1 = 2j$ and $q - 1 = 2k$. Then $(p-1, q-1) = (2j, 2k) = 2(j, k)$. We can see that there will be a factor of $2$ at a minimum when number are even. LCM is given by $lcm(p - 1, q - 1) = \frac{(p-1)(q-1)}{gcd(p-1, q-1)}$, which means $m = lcm(p-1, q-1)$ equals $(p-1)(q-1)$ only when $gcd(p-1, q-1) = 1$. But here we have $gcd > 1$ and hence $m = lcm(p-1, q-1) < (p-1)(q-1)$
				\end{flushleft}
				\item $a^m \equiv 1 \mod (p-1)(q-1)$
				\begin{flushleft}
					\textbf{\textit{Answer:}}
				\end{flushleft}
			\end{enumerate}
			\item Hence explain why numbers of the form n have no primitive roots.
			\begin{flushleft}
				\textbf{\textit{Answer:}} Suppose $n = p.q$, where $p$ and $q$ are primes has primitive roots. This means there exists $a \in (\mathbb{Z}/pq\mathbb{Z})^\times$ such that $ord_{pq}(a)$ will be $m = \phi(n) = \phi(p.q) = (p - 1).(q - 1)$. Also $gcd(a, pq) = 1$.
				\bigbreak
				
				Also, 
				\begin{align}
					& a^m \equiv 1\tpmod{pq} \\
					& \Longleftrightarrow a^m \equiv 1\tpmod p, a^m \equiv 1\tpmod q (\text{By Chinese Remainder Theorem})\\ 
					& \Longleftrightarrow m \equiv 0\tpmod{p-1}, m \equiv 0\tpmod{q-1}\\
					&(\text{Because by Fermat\'s Little Theorem, $a^{p-1} \equiv 1 \tpmod{p}$ and $a^{q-1} \equiv 1\tpmod{q}$})\nonumber\\
					& \Longleftrightarrow (p-1)|m, (q-1)|m \\
					& \Longleftrightarrow lcm(p-1, q-1)|m
				\end{align}
				This means that $ord_p(a) = lcm(p-1, q-1) < (p-1)(q-1)$ as we have seen in \ref{item:Q3biii} and it's a contradiction from our initial assumption that  $n = p.q$ has primitive roots.
			\end{flushleft}
			
			\item Show that all numbers of the form $n = pq$ (p and q both odd primes) in your range are included in set B.
			\begin{flushleft}
				\textbf{\textit{Answer:}}
				\begin{lstlisting}[
					language=json,
					firstnumber=1,
					caption={List of Numbers Without Primitive Roots},
					label={lst:nums-without-primitive-roots}
					]
					{
						"Numbers Without Primitive Roots (B)": [
						"600","602","603","604","605","606","608","609",
						"610","611","612","615","616","618","620","621",
						"623","624","627","628","629","630","632","633",
						"635","636","637","638","639","640","642","644",
						"645","646","648","649","650","651","652","654",
						"655","656","657","658","660","663","664","665",
						"666","667","668","669","670","671","672","675",
						"676","678","679","680","681","682","684","685",
						"687","688","689","690","692","693","695","696",
						"697","699","700","702","703","704","705","707",
						"708","710","711","712","713","714","715","716",
						"717","720","721","723","724","725","726","728",
						"730","731","732","735","736","737","738","740",
						"741","742","744","745","747","748","749","750"
						]
					}
				\end{lstlisting}
				
				\medbreak
				And the below table shows the list of numbers of th form $p.q$ in the range $600 \le n \le 750$, which is a subset of the set $B$ above \ref{lst:nums-without-primitive-roots}.
				\begin{table}[H]
					\centering
					\begin{tabular}{|l|l|l|l|}
						\hline
						Number & Form & Number & Form\\ \hline
						611 & $13^1\times47^1$ & 687 & $3^1\times229^1$ \\ \hline
						623 & $7^1\times89^1$ & 689 & $13^1\times53^1$ \\ \hline
						629 & $17^1\times37^1$ & 695 & $5^1\times139^1$ \\ \hline
						633 & $3^1\times211^1$ & 697 & $17^1\times41^1$ \\ \hline
						635 & $5^1\times127^1$ & 699 & $3^1\times233^1$ \\ \hline
						649 & $11^1\times59^1$ & 703 & $19^1\times37^1$ \\ \hline
						655 & $5^1\times131^1$ & 707 & $7^1\times101^1$ \\ \hline
						667 & $23^1\times29^1$ & 713 & $23^1\times31^1$ \\ \hline
						669 & $3^1\times223^1$ & 717 & $3^1\times239^1$ \\ \hline
						671 & $11^1\times61^1$ & 721 & $7^1\times103^1$ \\ \hline
						679 & $7^1\times97^1$ & 723 & $3^1\times241^1$ \\ \hline
						681 & $3^1\times227^1$ & 731 & $17^1\times43^1$ \\ \hline
						685 & $5^1\times137^1$ & 737 & $11^1\times67^1$ \\ \hline
						-   & -                & 745 & $5^1\times149^1$ \\ \hline
						-   & -                & 749 & $7^1\times107^1$ \\ \hline
					\end{tabular}
					\caption{Numbers of the form $p.q$}
					\label{table:number-p.q-without-primitive-roots}
				\end{table}
			\end{flushleft}
			
		\end{enumerate}
		\item Use the BabyStepsGiantSteps algorithm to find discrete logarithms x of b mod n for the primitive root a for each of the two examples assigned to you in the table below. Verify that your answer is correct by calculating $a^x \mod m$ by hand using the method of modular exponentiation.
		\begin{flushleft}
			\textit{Note: Somehow I couldn't make it work the Baby Steps Giant Steps Algorithm as we learned in the class. I checked the Wikipedia and it's the same as in the class. I do not know where did it go wrong. I was getting a smaller value that expected. So I followed the steps from some Youtube videos(\href{https://www.youtube.com/watch?v=007MVsELvQw}{Video1}, \href{https://www.youtube.com/watch?v=Cj_j_mX0qY0}{Video2}). The steps are similar with some minor variations in the values we calculate. Hope that's fine.}
			\bigbreak
			\begin{enumerate}
			\item \label{lbl:verify-pollards-rho}\textbf{\textit{Answer:}} Given $a = 21, b = 47, n = 71$. We want to solve for $t$ in the congruence: $21^t \equiv 47\tpmod{71}$\\ \medbreak
			We have $\phi(71) = 70$. 
			\medbreak
			\begin{enumerate}[Step 1.]
			\item Set m = $\lceil\sqrt{71}\rceil = 9$
			
			\item Calculating $a^{mj}\tpmod{71}; 0 \le j < m$
			\begin{table}[H]
				\begin{adjustbox}{scale=0.9, center}
				\begin{tabular}{ ||l|l||l|l||l|l|| }
					\hline
					j & $a^{mj}\tpmod{71}$ & j & $a^{mj}\tpmod{71}$ & j & $a^{mj}\tpmod{71}$\\
					\hline
					0 & $21^{9.0} = 1$ & 3 & $21^{9.3} = 35$ & 6 & $21^{9.6} = 18$\Tstrut\\ 
					\hline 
					1 & $21^{9.1} = 42$ & 4 & $21^{9.4} = 50$ & 7 & $21^{9.7} = 46$\Tstrut\\ 
					\hline 
					2 & $21^{9.2} = 60$ & 5 & $21^{9.5} = 41$ & 8 & $21^{9.8} = 15$\Tstrut\\ 
					\hline 
				\end{tabular}
				\end{adjustbox}
			\end{table}
			\bigbreak
			\item Solve for $b.a^{-i}; 0 \le i < m$
			\begin{table}[H]
				\begin{adjustbox}{scale=1, center}
					\begin{tabular}{ |l|l||l|l| }
						\hline
						i & $b.a^{-i}\tpmod{71}$ & i & $b.a^{-i}\tpmod{71}$ \\
						\hline
						0 & $47.21^{0} = 47$ & 4 & $47.21^{-4} = 47.21^{66} = 69$ \Tstrut\\ 
						\hline
						1 & $47.21^{-1} = 47.21^{69} = 9$ & 5 & $47.21^{-5} = 47.21^{65} = 54$ \Tstrut\\ 
						\hline
						2 & $47.21^{-2} = 47.21^{68} = 41$ & 6 & $47.21^{-6} = 47.21^{64} = 33$ \Tstrut\\ 
						\hline
						3 & $47.21^{-3} = 47.21^{67} = 29$ & 7 & $47.21^{-7} = 47.21^{63} = 32$ \Tstrut\\ 
						\hline
						- & - & 8 & $47.21^{-8} = 47.21^{62} = 59$ \Tstrut\\ 
						\hline
					\end{tabular}
				\end{adjustbox}
			\end{table}
			\item We found a collision in both the tables for $value = 41$ where $i = 2$ and $mj = 9 \times 5$
			\item We calculate $t = (mj + i)\tpmod{71} = (9 \times 5 + 2)\tpmod{71} \equiv 47\tpmod{71}$\\
			$\implies 21^{47} \equiv 47\tpmod{71}$
			\item Verifying the answer using Fast Modular Exponentiation:
			\begin{align*}
				& 21^{2} \equiv 15\tpmod{71}  \\
				& \therefore 21^4 = (21^2)^2 = 15^2\tpmod{71}\equiv12\tpmod{71} \\
				& \implies 21^8 = (21^4)^2 = 12^2 \equiv 2\tpmod{71} \\
				& \implies 21^{16} = (21^8)^2 = 2^2 \equiv 4 \tpmod{71} \\
				& \implies 21^{32} = (21^{16})^2 = 4^2 \equiv 16\tpmod{71} \\
				& \implies 21^{40} = 21^{32}\times21^{8}\tpmod{71} \equiv 16\times2\tpmod{71}\equiv32\tpmod{71} \\ 
				& \text{We will calculate now }21^7\tpmod{71} \text{using the below steps:}\\
				& \text{The binary representation for }7 = [111]  \sim [d_2d_1d_0] \\
				& \text{Let } a = 1 \text{ and } s = 21 \\
				& k = 0: \text{Since } d_k = 1, a = a \times s = 21\tpmod{71}, s = s^2 = 15\tpmod{71} \\
				& k = 1: \text{Since } d_k = 1, a = a \times s = 31\tpmod{71}, s = s^2 = 12\tpmod{71} \\
				& k = 2: \text{Since } d_k = 1, a = a \times s = 17\tpmod{71}, \\
				& \implies 21^7 \equiv 17 \tpmod{71} \\
				& \therefore 21^{47} = 21^{40} \times 21^{7} = 32 \times 17 \equiv 47\tpmod{71} \text{ and hence the answer}
 			\end{align*}
		\end{enumerate}
		
		\item \textbf{\textit{Answer:}} Given $a = 26, b = 24, n = 53$. We want to solve for $t$ in the congruence: $26^t \equiv 24\tpmod{53}$\\ 
		 
		\medbreak
		\begin{enumerate}[Step 1.]
			\item Set m = $\lceil\sqrt{53}\rceil = 8$
			\item --
			\item Calculating $a^{mj}\tpmod{53}; 0 \le j < m$ \& Solve for $b.a^{-i}\tpmod{53}; 0 \le i < m$ (Step 2 and 3 tables below side-by-side)
			\medbreak
			$26^{-1} \equiv27\tpmod{53}$
			\begin{table}[H]
				\parbox{.50\linewidth}{
					\centering
					\begin{tabular}{|l|l|}
						\hline
						j & $a^{mj}\tpmod{71}$ \Tstrut\\
						\hline
						0 & $26^{8.0} \equiv 1$ \Tstrut\\
						\hline
						1 & $26^{8.1} \equiv 47$ \Tstrut\\ 
						\hline
						2 & $26^{8.2} \equiv 36$ \Tstrut\\ 
						\hline
						3 & $26^{8.3} \equiv 49$ \Tstrut\\ 
						\hline
						4 & $26^{8.4} \equiv 24$ \Tstrut\\ 
						\hline
						5 & $26^{8.5} \equiv 15$ \Tstrut\\ 
						\hline
						6 & $26^{8.6} \equiv 16$ \Tstrut\\ 
						\hline
						7 & $26^{8.7} \equiv 10$ \Tstrut\\ 
						\hline
					\end{tabular}
					\caption{Step 2}
				}
				\hfill
				\parbox{.50\linewidth}{
					\centering
					\begin{tabular}{|l|l|}
						\hline
						i & $b.a^{-i}\tpmod{53}$ \Tstrut\\
						\hline
						0 & $24.26^0 = 24.26^{52} = 26$ \Tstrut\\ 
						\hline
						1 & $24.26^{-1} = 24.26^{51} = 5$ \Tstrut\\ 
						\hline
						2 & $24.26^{-2} = 24.26^{50} = 43$ \Tstrut\\
						\hline
						3 & $24.26^{-3} = 24.26^{49} = 20$ \Tstrut\\
						 \hline
						4 & $24.26^{-4} = 24.26^{48} = 13$ \Tstrut\\
						\hline
						5 & $24.26^{-5} = 24.26^{47} = 27$ \Tstrut\\
						\hline
						6 & $24.26^{-6} = 24.26^{46} = 52$ \Tstrut\\
						\hline
						7 & $24.26^{-7} = 24.26^{45} = 2$ \Tstrut\\
						\hline
						8 & $24.26^{-8} = 24.26^{44} = 49$ \Tstrut\\
						\hline
					\end{tabular}
					\caption{Step 3}}
			\end{table}
			\item We found a collision in both the tables for $value = 49$ where $i = 8$ and $mj = 8 \times 3$
			\item We calculate $t = (mj + i)\tpmod{53} = (8 \times 3 + 8)\tpmod{53} \equiv 32\tpmod{53}$\\
			$\implies 26^{32} \equiv 24\tpmod{53}$
			\item Verifying the answer using Fast Modular Exponentiation:
			\begin{align*}
				& 26^2 \equiv 40\tpmod{53} \\
				& \therefore 26^4 = (26^2)^2 = 40^2 \equiv 10\tpmod{53} \\
				& \implies 26^{16} = (26^4)^4 = 10^4 \equiv 36\tpmod{53} \\
				& \implies 26^{32} = (26^{16})^2 = 36^2 \equiv 24\tpmod{53} \text{ and hence the answer}
			\end{align*}
		\end{enumerate}	
		\end{enumerate}
		\end{flushleft}
		\item Use the Pohlig Helmann algorithm to find in the cyclic group of order n with the generating element a for both the examples assigned to you below. Verify your answer in Maple.
		\begin{flushleft}
			\begin{enumerate}
				\item \textbf{\textit{Answer:}} $a = x^{11}, b = x^{41}, n = 343$ \\ \bigbreak
				$n = 343 = 7^3$\\
				We will write G as $G = \{x^i | 0 \le i \le 342, x^{342} = 1\}$. Also $x^{11}$ generates $G$ as $(11, 343) = 1$.\\
				
				So we set $p = 7, e = 3, g = x^{11}, h = x^{41}$
				\begin{enumerate}[Step 1.]
					\item Setting $x_0 = 0$ and let $n = p^e, p^{e-1} = p^2 = 49 $\\
					When $k = 0$:\\
					$s = g^{p^{e-1}} = g^{n/p} = (x^{11})^{49} = x^{539} = x^{196}$\\
					$h_0 = (g^{-x_0} \times h)^{p^{e-1}} = h^{49} = (x^{41})^{49} = x^{294}$ \\
					Since $p = 7$, test for $d_0 \in \{0, 1, 2, 3, 4, 5, 6\}$ satisfying $s^{d_0} = h_0$\\
					$\therefore$, for $d_0 = 5$, we have $s^{d_0} = h_0$, so $d_0 = 5$\\
					$x_1 = x_0 + p^0.d_0 = 0 + 1.5 = 5$
					\item When $k = 1$ we have $x_1 = 5, p^{e-2} = p^1 = 7$\\
					$h_1 = (g^{-x_1}\times h)^{p^{e-2}} = (g^{-5}\times h)^7 = (x^{-55} \times x^{41})^7 = x^{-98} = x^{245}$\\
					Searching for $d_1 \in \{0, 1, 2, 3, 4, 5, 6\}$ satisfying $s^{d_1} = h_1$\\
					$\therefore, r = 3$ satisfies the condition. So $d_1 = 3$\\
					$x_2 = x_1 + p^1.d_1 = 5 + 7 \times 3 = 26$\\
					\item When $k = 2$, we have $x_2 = 26, p^{e-3} = 1$\\
					$h_2 = (g^{-x_2} \times h)p^{e-3} = (g^{-26} \times h)^1 = x^{-286} \times x^{41} = x^{-245} = x^{98}$\\
					Searching for $d_2$, we get $d_2 = 4$\\
					$x_3 = x_2 + p^2.d_2 = 26 + 49 \times 4 = 222$  
				\end{enumerate}
				$x = 222$ is the logarithm we wanted.
				\bigbreak
				Below is the Maple Verification result:
				\begin{mdframed}
					
					\begin{Maple Normal}
						{$ \displaystyle G ={\{x^{i}| 0\le i \le 342,x^{342}=1\}} $}
					\end{Maple Normal}
					\begin{Maple Normal}
						{$ \displaystyle \gcd (11,343)=1,x^{11}\mathit{generates} \mathit{the} \mathit{Group} . $}
					\end{Maple Normal}
					\mapleinput
					{$ \displaystyle p \coloneqq 7;e \coloneqq 3;g \coloneqq x^{11};h \coloneqq x^{41}; $}
					
					% \mapleresult
					\begin{dmath*}
						p \coloneqq 7
					\end{dmath*}
					\vspace{-\bigskipamount}
					% \mapleresult
					\begin{dmath*}
						e \coloneqq 3
					\end{dmath*}
					\vspace{-\bigskipamount}
					% \mapleresult
					\begin{dmath*}
						g \coloneqq x^{11}
					\end{dmath*}
					\vspace{-\bigskipamount}
					% \mapleresult
					\begin{dmath}\label{(1)}
						h \coloneqq x^{41}
					\end{dmath}
					\begin{Maple Normal}
						Step1:
					\end{Maple Normal}
					\mapleinput
					{$ \displaystyle \mathit{x0} \coloneqq 0; $}
					
					% \mapleresult
					\begin{dmath}\label{(2)}
						\mathit{x0} \coloneqq 0
					\end{dmath}
					\mapleinput
					{$ \displaystyle s \coloneqq x^{11\cdot 49\,\mod \,343};\,\mathit{h0} \coloneqq x^{41\cdot 49\,\mod \,343}; $}
					
					% \mapleresult
					\begin{dmath*}
						s \coloneqq x^{196}
					\end{dmath*}
					\vspace{-\bigskipamount}
					% \mapleresult
					\begin{dmath}\label{(3)}
						\mathit{h0} \coloneqq x^{294}
					\end{dmath}
					\begin{Maple Normal}
						{$ \displaystyle \mathit{Searching} \boldsymbol{\mathrm{for}}\mathit{d0} ;\mathit{d0} =5\mathit{satisfies} s^{\mathit{d0}}=\mathit{h0}  $}
					\end{Maple Normal}
					\mapleinput
					{$ \displaystyle \mathit{d0} \coloneqq 5; $}
					
					% \mapleresult
					\begin{dmath}\label{(4)}
						\mathit{d0} \coloneqq 5
					\end{dmath}
					\mapleinput
					{$ \displaystyle x^{196\cdot 5\,\mod \,343}; $}
					
					% \mapleresult
					\begin{dmath}\label{(5)}
						x^{294}
					\end{dmath}
					\mapleinput
					{$ \displaystyle \mathit{x1} \coloneqq \mathit{x0} +p^{0}\cdot \mathit{d0} ; $}
					
					% \mapleresult
					\begin{dmath}\label{(6)}
						\mathit{x1} \coloneqq 5
					\end{dmath}
					\begin{Maple Normal}
						Step 2:
					\end{Maple Normal}
					\mapleinput
					{$ \displaystyle \mathit{h1} \coloneqq x^{\mathit{(-55+41)}\cdot 7\,\mod \,343}; $}
					
					% \mapleresult
					\begin{dmath}\label{(7)}
						\mathit{h1} \coloneqq x^{245}
					\end{dmath}
					\begin{Maple Normal}
						{$ \displaystyle \mathit{Searching} \boldsymbol{\mathrm{for}}\mathit{d1} ;\mathit{d1} =3\mathit{satisfies} s^{\mathit{d1}}=\mathit{h1}  $}
					\end{Maple Normal}
					\mapleinput
					{$ \displaystyle \mathit{d1} \coloneqq 3; $}
					
					% \mapleresult
					\begin{dmath}\label{(8)}
						\mathit{d1} \coloneqq 3
					\end{dmath}
					\mapleinput
					{$ \displaystyle x^{196\cdot 3\,\mod \,343}; $}
					
					% \mapleresult
					\begin{dmath}\label{(9)}
						x^{245}
					\end{dmath}
					\mapleinput
					{$ \displaystyle \mathit{x2} \coloneqq \mathit{x1} +p^{1}\cdot \mathit{d1} ; $}
					
					% \mapleresult
					\begin{dmath}\label{(10)}
						\mathit{x2} \coloneqq 26
					\end{dmath}
					\begin{Maple Normal}
						Step 3:
					\end{Maple Normal}
					\mapleinput
					{$ \displaystyle \mathit{h2} \coloneqq x^{-286+41\,\mod \,343}; $}
					
					% \mapleresult
					\begin{dmath}\label{(11)}
						\mathit{h2} \coloneqq x^{98}
					\end{dmath}
					\begin{Maple Normal}
						{$ \displaystyle \mathit{Searching} \boldsymbol{\mathrm{for}}\mathit{d2} ;\mathit{d2} =4\mathit{satisfies} s^{\mathit{d2}}=\mathit{h2}  $}
					\end{Maple Normal}
					\mapleinput
					{$ \displaystyle \mathit{d2} \coloneqq 4; $}
					
					% \mapleresult
					\begin{dmath}\label{(12)}
						\mathit{d2} \coloneqq 4
					\end{dmath}
					\mapleinput
					{$ \displaystyle x^{196\cdot 4\,\mod \,343}; $}
					
					% \mapleresult
					\begin{dmath}\label{(13)}
						x^{98}
					\end{dmath}
					\mapleinput
					{$ \displaystyle \mathit{x3} \coloneqq \mathit{x2} +p^{2}\cdot \mathit{d2} ; $}
					
					% \mapleresult
					\begin{dmath}\label{(14)}
						\mathit{x3} \coloneqq 222
					\end{dmath}
					\begin{Maple Normal}
						{$ \displaystyle \mathit{x3} \mathit{is} \mathit{our} \mathit{logarithm}  $}
					\end{Maple Normal}
				\end{mdframed}
				
				\bigbreak
				\item \textbf{\textit{Answer:}} $a = x^{13}, b = x^{157}, n = 3267$ \bigbreak
				$n = 3267 = 3^3 \times 11^2 = 27 \times 121$\\
				We will write G as $G = \{x^i | 0 \le i \le 3266, x^{342} = 1\}$. Also $x^{13}$ generates $G$ as $(13, 3267) = 1$.\\
				
				\begin{enumerate}[Step 1.]
					\item 
					\begin{align*}
						& g1 = g^{121} = x^{13 \times 121 \tpmod{27}} = x^7 \\
						& h1 = h^{121} = x^{157 \times 121 \tpmod{27}} = x^{16}\\					
					\end{align*}
					We will need to find the logarithm of $h1 = x^{16}$ in the cyclic group of order $27$ generated by $g1 = x^7$. With some trial and error, we get $log x^{16} = 10$, i.e., $x^{7 \time 10\tpmod{27}} = x^{16}$. Hence we get the below congruence:
					
					\begin{equation}\label{eqn:cng1}
						x \equiv 10 \tpmod{27}
					\end{equation}
					\item 
					\begin{align*}
						g2 = g^{27} = x^{13 \times 27 \tpmod{121}} = x^{109} \\
						h2 = h^{27} = x^{157 \times 27 \tpmod{121}} = x^{4} \\
					\end{align*}
					Let's find the logarithm of $h2 = x^4$ in the cyclic group of order $121$ generated by $g2 = x^{109}$. With some trial and error, we get $log x^4 = 40$, i.e., $x^{109 \times 40 \tpmod{121}} = 4$. We get the following congruence: 
					\begin{equation}\label{eqn:cng2}
						x \equiv 40 \tpmod{121}
					\end{equation}
					\item We will now need to solve the congruences \ref{eqn:cng1} and \ref{eqn:cng2}. We will employ Chinese Remainder Theorem for that. Suppose we have a system of congruences as below:
					\begin{equation}
						\begin{cases}
							x & \equiv b_1 \tpmod{n_1}\\
							x & \equiv b_2 \tpmod{n_2}\\
							x & \equiv b_3 \tpmod{n_3}\\
							  &. \\
							  &. \\
							  &. \\
							x & \equiv b_k \tpmod{n_k}\\
						\end{cases}       
					\end{equation}
					CRT states that the above congruence has a unique modulo $N = n_1.n_2.n_3...n_k$ solution if each $n_i$ are pairwise coprime and is given by:
					\begin{align*}
						x &= \sum_{i=1}^{k} b_ie_i(N/n_i) \\
						&\text{where } e_i = (N/n_i)^{-1}\tpmod{n_i} \\
					\end{align*}
					Restating our equations below:
					\begin{equation}
						\begin{cases}
							x & \equiv 10 \tpmod{27}\\
							x & \equiv 40 \tpmod{121}\\
						\end{cases}       
					\end{equation}
					\begin{align*}
						&n_1 = 27, n_2 = 121\\
						&N = n = 27 \time 121 = 3267 \\
						&b_1 = 10, b_2 = 40 \\
						&e_1 = 121^{-1} \tpmod{27} = 25\\
						&e_2 = 27^{-1} \tpmod{121} = 9\\
						&\therefore x = 10 \times 25 \times 121 + 40 \times 9 \times 27 = 766\tpmod{3267} 
					\end{align*}
				\end{enumerate}
				$x = 766$ is the logarithm we wanted.
				\bigbreak
				Below is the Maple Verification result:
				\begin{mdframed}
					\begin{Maple Normal}
						{$ \displaystyle \textit{find\_log} \coloneqq \boldsymbol{\mathrm{proc}}(n ,a ,m)
							\\
							\boldsymbol{\mathrm{description}}\text{``Find log of a''};
							\\
							\boldsymbol{\mathrm{for}}i \boldsymbol{\mathrm{from}}1\boldsymbol{\mathrm{to}}n 
							\\
							\boldsymbol{\mathrm{do}}
							\\
							\boldsymbol{\mathrm{if}}a \cdot i \boldsymbol{\mod}n =m 
							\\
							\boldsymbol{\mathrm{then}}
							\\
							\boldsymbol{\mathrm{return}}i ;
							\\
							\boldsymbol{\mathrm{fi}};
							\\
							\boldsymbol{\mathrm{end}}\boldsymbol{\mathrm{do}};
							\\
							\boldsymbol{\mathrm{end}}\boldsymbol{\mathrm{proc}}; $}
					\end{Maple Normal}
					
					{$ \displaystyle  $}\begin{lstlisting}
						> 
					\end{lstlisting}
					\begin{Maple Normal}
						
					\end{Maple Normal}
					\mapleinput
					{$ \displaystyle \mathit{ifactor} (3267) $}
					
					% \mapleresult
					\begin{dmath}\label{(1)}
						\left(3\right)^{3} \left(11\right)^{2}
					\end{dmath}
					
					{$ \displaystyle n \coloneqq 3^{3}\cdot 11^{2}; $}
					{$ \displaystyle \, $}% \mapleresult
					\begin{dmath}\label{(2)}
						n \coloneqq 3267
					\end{dmath}
					\begin{Maple Normal}
						g = x^13, h = x^157, n = 3267; <g> generatees the group of order 3267;
						
						
						Steps 1:
						
						
					\end{Maple Normal}
					\mapleinput
					{$ \displaystyle \mathit{g1} \coloneqq x^{13\cdot 11^{2}\mod \,27}; $}
					
					% \mapleresult
					\begin{dmath}\label{(3)}
						\mathit{g1} \coloneqq x^{7}
					\end{dmath}
					\mapleinput
					{$ \displaystyle \mathit{h1} \coloneqq x^{157\cdot 11^{2}\mod \,27}; $}
					
					% \mapleresult
					\begin{dmath}\label{(4)}
						\mathit{h1} \coloneqq x^{16}
					\end{dmath}
					\begin{Maple Normal}
						We need to find the log of h1 = x^16 in the cyclic group of order 27 generated by g1 = x^7. By trial and error, we get log(h1) = 10
					\end{Maple Normal}
					\begin{Maple Normal}
						
					\end{Maple Normal}
					
					{$ \displaystyle \textit{find\_log} (27,7,16); $}\begin{lstlisting}
						> 
					\end{lstlisting}
					% \mapleresult
					\begin{dmath}\label{(5)}
						10
					\end{dmath}
					\begin{Maple Normal}
						So our first congruence is x1 = 10 mod 27 --- (1)
					\end{Maple Normal}
					\begin{Maple Normal}
						
					\end{Maple Normal}
					\begin{Maple Normal}
						Step 2:
					\end{Maple Normal}
					\begin{Maple Normal}
						
					\end{Maple Normal}
					\mapleinput
					{$ \displaystyle \mathit{g2} \coloneqq x^{13\cdot 3^{3}\mod \,121}; $}
					
					% \mapleresult
					\begin{dmath}\label{(6)}
						\mathit{g2} \coloneqq x^{109}
					\end{dmath}
					\mapleinput
					{$ \displaystyle \mathit{h2} \coloneqq x^{157\cdot 3^{3}\mod \,121}; $}
					
					% \mapleresult
					\begin{dmath}\label{(7)}
						\mathit{h2} \coloneqq x^{4}
					\end{dmath}
					\begin{Maple Normal}
						We need to find the log of h2 = x^4 in the cyclic group of order 121 generated by g2 = x^109; using the proc find\_log above, it is = 118
					\end{Maple Normal}
					\mapleinput
					{$ \displaystyle \textit{find\_log} (121,109,4); $}
					
					% \mapleresult
					\begin{dmath}\label{(8)}
						40
					\end{dmath}
					\begin{Maple Normal}
						We get our second congruence as: x2 = 40 mod 121 --- (2)
					\end{Maple Normal}
					\begin{Maple Normal}
						
					\end{Maple Normal}
					\begin{Maple Normal}
						Hence  we need to find the unique solution to x = 10 mod 27, and x = 40 mod 121 using Chinese Remainder Theorem.
					\end{Maple Normal}
					\begin{Maple Normal}
						
					\end{Maple Normal}
					\mapleinput
					{$ \displaystyle \mathit{with} (\mathit{NumberTheory}); $}
					
					% \mapleresult
					\begin{dmath}\label{(9)}
						\left[\mathit{AreCoprime} ,\mathit{CalkinWilfSequence} ,\mathit{CarmichaelLambda} ,
						\\
						\mathit{ChineseRemainder} ,\mathit{ContinuedFraction} ,
						\\
						\mathit{ContinuedFractionPolynomial} ,\mathit{CyclotomicPolynomial} ,
						\\
						\mathit{Divisors} ,\mathit{FactorNormEuclidean} ,\mathit{HomogeneousDiophantine} ,
						\\
						\mathit{ImaginaryUnit} ,\mathit{InhomogeneousDiophantine} ,\mathit{IntegralBasis} ,
						\\
						\mathit{InverseTotient} ,\mathit{IsCyclotomicPolynomial} ,\mathit{IsMersenne} ,
						\\
						\mathit{IsSquareFree} ,\mathit{IthFermat} ,\mathit{IthMersenne} ,\mathit{JacobiSymbol} ,
						\\
						\mathit{JordanTotient} ,\mathit{KroneckerSymbol} ,\mathit{Landau} ,\mathit{LargestNthPower} ,
						\\
						\mathit{LegendreSymbol} ,M\ddot{o}\mathit{bius} ,\mathit{ModExtendedGCD} ,\mathit{ModularLog} ,
						\\
						\mathit{ModularRoot} ,\mathit{ModularSquareRoot} ,\mathit{Moebius} ,
						\\
						\mathit{MultiplicativeOrder} ,M\ddot{o}\mathit{bius} ,\mathit{NearestLatticePoint} ,
						\\
						\mathit{NextSafePrime} ,\mathit{NumberOfIrreduciblePolynomials} ,
						\\
						\mathit{NumberOfPrimeFactors} ,\Omega ,\Phi ,\mathit{PrimeCounting} ,\mathit{PrimeFactors} ,
						\\
						\mathit{PrimitiveRoot} ,\mathit{PseudoPrimitiveRoot} ,\mathit{QuadraticResidue} ,
						\\
						\mathit{Radical} ,\mathit{RepeatingDecimal} ,\mathit{RootsOfUnity} ,
						\\
						\mathit{SimplestRational} ,\mathit{SumOfDivisors} ,\mathit{SumOfSquares} ,\mathit{ThueSolve} ,
						\\
						\mathit{Totient} ,\lambda ,\mu ,\phi ,\mathrm{pi},\sigma ,\tau ,\varphi \right]
					\end{dmath}
					\begin{Maple Normal}
						
					\end{Maple Normal}
					\mapleinput
					{$ \displaystyle \mathit{ChineseRemainder} ([10,40],[27,121]); $}
					
					% \mapleresult
					\begin{dmath}\label{(10)}
						766
					\end{dmath}
				\end{mdframed}
			\end{enumerate}
		\end{flushleft}
		\item Use the Pollard Rho method to verify your answer to the first example you were allocated in Question 4. 
		
		\begin{table}[H]
			\begin{adjustbox}{scale=0.9,center}
				\begin{tabular}{ |p{2cm}|p{2cm}|p{2cm}|p{2cm}|p{4cm}| }
					\hline
					Name & b & n & a & Method \\
					\hline
					Ajeesh & $47$   & 71   & $21$   & BabyStepGiantStep \\
					Ajeesh & $24$   & 53   & $26$   & BabyStepGiantStep  \\
					Ajeesh & $x^{41}$ & 343  & $x^{11}$ & Pohlig Hellmen \\
					Ajeesh & $x^{157}$& 3267 & $x^{13}$ & Pohlig Hellmen  \\
					\hline
				\end{tabular}
			\end{adjustbox}
			\caption{Table Listing the Problems Allocation to Individuals}
			\label{table:assignment-tab}
		\end{table}
		
		\begin{flushleft}
			\textbf{\textit{Answer:}} The below table shows the execution of the Pollard Rho algorithm to generate the data table:
			\begin{table}[H]
				\begin{adjustbox}{scale=0.9,center}
				\begin{tabular}{ |p{2cm}|p{2cm}|p{2cm}|p{2cm}||p{2cm}|p{2cm}|p{2cm}| }
				\hline
				\multicolumn{7}{|c|}{Pollard Rho Execution Data} \\
				\hline
				i & x1 & a1 & b1 & x2 & a2 & b2 \\ \hline
				1 & 21 & 1 & 0 & 15 & 2 & 0 \\ \hline
				2 & 15 & 2 & 0 & 2 & 8 & 0 \\ \hline
				3 & 12 & 4 & 0 & 16 & 8 & 2 \\ \hline
				4 & 2 & 8 & 0 & 27 & 10 & 2 \\ \hline
				5 & 23 & 8 & 1 & 44 & 21 & 4 \\ \hline
				6 & 16 & 8 & 2 & 10 & 42 & 10 \\ \hline
				7 & 52 & 9 & 2 & 1 & 43 & 11 \\ \hline
				8 & 27 & 10 & 2 & 15 & 18 & 22 \\ \hline
				9 & 19 & 20 & 4 & 2 & 2 & 18 \\ \hline
				10 & 44 & 21 & 4 & 16 & 2 & 20 \\ \hline
				11 & 9 & 21 & 5 & 27 & 4 & 20 \\ \hline
				12 & 10 & 42 & 10 & 44 & 9 & 40 \\ \hline
				13 & 68 & 43 & 10 & 10 & 18 & 12 \\ \hline
				14 & 1 & 43 & 11 & 1 & 19 & 13 \\ \hline
				\end{tabular}
				\end{adjustbox}
				\caption{Pollard Rho Execution Data}
				\label{table:pollards-rho}
			\end{table}
			
			From the table, we can see that $x1 = x2$ at the $14^{th}$ iteration. The corresponding a1, a2, b1, b2 values are: $i = 14$, $a1 = 43$, $a2 = 19$, $b1 = 11$, $b2 = 13$.
			
			Let $t$ be the logarithm of $b$. Calculation of the logarithm is below:
			\begin{align}
				&(b2 - b1).t \equiv (a1 - a2)\tpmod{p-1} \nonumber\\
				&\implies (13 - 11).t \equiv (43 - 19)\tpmod{70} \nonumber\\
				&\implies 2.t \equiv 24\tpmod{70}\\
				&\text{Let }d = gcd(2, 70) = 2 \nonumber\\
				&\text{Divide Eqn. (34) by d, we get: } t \equiv 12 \tpmod{35} \nonumber\\
				&\text{Since gcd = 2, there are two solution to Eqn. (35) and the solutions are: }  \nonumber\\
				&\{12, 12 + \frac{70}{2}\} = \{12, 47\}
			\end{align}
			When $t = 47$ our congruence $21^t \equiv 47\tpmod{71}$ is satisfied and hence the logarithm of $47$ is $47$. Thus we have verified solution of the problem in \ref{lbl:verify-pollards-rho}.
		\end{flushleft}
	\end{enumerate}
	
	\begin{thebibliography}{unsrt}
		
		\bibitem{Modular_Mathematics}
		C R Jordan \& D A Jordan \emph{MODULAR MATHEMATICS Groups }.
		
		\bibitem{gt_solutions}
		Dr. Ben Fairbairn \emph{GROUP THEORY Solutions to Exercises}.
		
		\bibitem{online_ref_1}
		\emph{https://github.com/Ssophoclis/AKS-algorithm/blob/master/AKS.py}
		
	\end{thebibliography}
	
\end{document}