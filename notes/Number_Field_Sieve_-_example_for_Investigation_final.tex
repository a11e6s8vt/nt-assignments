%% Created by Maple 2023.2, Mac OS X
%% Source Worksheet: Number_Field_Sieve_-_example_for_Investigation_final.mw
%% Generated: Tue Jan 09 19:34:52 GMT 2024
\documentclass{article}
\usepackage{amssymb}
\usepackage{graphicx}
\usepackage{hyperref}
\usepackage{listings}
\usepackage{mathtools}
\usepackage{maple}
\usepackage[utf8]{inputenc}
\usepackage[svgnames]{xcolor}
\usepackage{amsmath}
\usepackage{breqn}
\usepackage{textcomp}
\begin{document}
\lstset{basicstyle=\ttfamily,breaklines=true,columns=flexible}
\pagestyle{empty}
\DefineParaStyle{Maple Bullet Item}
\DefineParaStyle{Maple Heading 1}
\DefineParaStyle{Maple Warning}
\DefineParaStyle{Maple Heading 4}
\DefineParaStyle{Maple Heading 2}
\DefineParaStyle{Maple Heading 3}
\DefineParaStyle{Maple Dash Item}
\DefineParaStyle{Maple Error}
\DefineParaStyle{Maple Title}
\DefineParaStyle{Maple Ordered List 1}
\DefineParaStyle{Maple Text Output}
\DefineParaStyle{Maple Ordered List 2}
\DefineParaStyle{Maple Ordered List 3}
\DefineParaStyle{Maple Normal}
\DefineParaStyle{Maple Ordered List 4}
\DefineParaStyle{Maple Ordered List 5}
\DefineCharStyle{Maple 2D Output}
\DefineCharStyle{Maple 2D Input}
\DefineCharStyle{Maple Maple Input}
\DefineCharStyle{Maple 2D Math}
\DefineCharStyle{Maple Hyperlink}
\mapleinput
{$ \displaystyle \texttt{>\,} \, $}

\begin{Maple Normal}
Here we look to break down an element of Q
{$ ((-2)^{\frac{1}{3}})\mathit{into} a \mathit{set} \mathit{of} \mathit{primes} \mathit{with} \mathit{relatively} \mathit{small} \mathit{norms} .\mathit{We} \mathit{need} \mathit{procedures}  $}for Norm, Multiplication, Inversion and Division in the field 
\end{Maple Normal}
\begin{Maple Normal}
The three roots of the equation z
{$ ^{3} $} = -2 are given below. If z1 is the real root then the two conjugate complex roots are z2 = wz and z3 = 
{$ w^{2} $}z where 
\end{Maple Normal}
\begin{Maple Normal}
w:= 1/2*(-1 +i
{$ *\mathrm{sqrt}(3)\right); $}
\end{Maple Normal}
\mapleinput
{$ \displaystyle \texttt{>\,} \mathit{solve} (z^{3}=-2,z); $}

% \mapleresult
\begin{dmath}\label{(1)}
-2^{\frac{1}{3}},\frac{2^{\frac{1}{3}}}{2}-\frac{\mathrm{I} \sqrt{3}\, 2^{\frac{1}{3}}}{2},\frac{2^{\frac{1}{3}}}{2}+\frac{\mathrm{I} \sqrt{3}\, 2^{\frac{1}{3}}}{2}
\end{dmath}
\mapleinput
{$ \displaystyle \texttt{>\,} \mathit{solve} (w^{2}+w +1,w); $}

% \mapleresult
\begin{dmath}\label{(2)}
-\frac{1}{2}+\frac{\mathrm{I} \sqrt{3}}{2},-\frac{1}{2}-\frac{\mathrm{I} \sqrt{3}}{2}
\end{dmath}
\begin{Maple Normal}
Each element of the ring has the form a + bz +cz^2 where a is an integer which we write as [a,b,c]. The norm of an element [a,b,c] is               (a + bz + cz^2)(a+bwz+cw^2)(a+bw^2z+cwz^2). Because z^3 = -2 and w^2+w+1 = 0 this simplifies as norm1 below.
\end{Maple Normal}
\begin{Maple Normal}

\end{Maple Normal}
\mapleinput
{$ \displaystyle \texttt{>\,} \mathit{norm1} \coloneqq \mathrm{proc}(a ,b ,c)\,\mathrm{global}\,k ;\,k \coloneqq a^{3}-2\cdot b^{3}+4\cdot c^{3}+6\cdot a \cdot b \cdot c ;\mathrm{end}; $}

% \mapleresult
\begin{dmath}\label{(3)}
\mathit{norm1} \coloneqq \boldsymbol{\mathrm{proc}}\left(a ,b ,c \right)\quad \boldsymbol{\mathrm{global}}\quad k ;\quad k \coloneqq a ^{3}-2*b ^{3}+4*c ^{3}+6*b *a *c \quad \boldsymbol{\textrm{end proc}}
\end{dmath}
\mapleinput
{$ \displaystyle \texttt{>\,} \mathit{norm1} (66,53,0); $}

% \mapleresult
\begin{dmath}\label{(4)}
-10258
\end{dmath}
\begin{Maple Normal}
The following procedures perform multiplication, inversion and division in our field
\end{Maple Normal}
\begin{Maple Normal}

\end{Maple Normal}
\mapleinput
{$ \displaystyle \texttt{>\,} \,\mathit{mult2} \coloneqq \mathrm{proc}(a ,b ,c ,d ,e ,f)\mathrm{global}\,\mathit{mul1} ,\mathit{mul2} ,\mathit{mul3} ;\, \,\mathit{mul1} \coloneqq a \cdot d -2\cdot b \cdot f -2\cdot c \cdot e ;
\\
 \,\mathit{mul2} \coloneqq a \cdot e +b \cdot d -2\cdot c \cdot f ;
\\
 \,\mathit{mul3} \coloneqq a \cdot f +b \cdot e +c \cdot d ;
\\
 \mathit{RETURN} (\mathit{mul1} ,\mathit{mul2} ,\mathit{mul3});\mathrm{end};\, $}

% \mapleresult
\begin{dmath}\label{(5)}
\mathit{mult2} \coloneqq \boldsymbol{\mathrm{proc}}\left(a ,b ,c ,d ,e ,f \right)\quad \boldsymbol{\mathrm{global}}\quad \mathit{mul1} ,\mathit{mul2} ,\mathit{mul3} ;\quad \mathit{mul1} \coloneqq a *d -2*b *f -2*c *e ;\quad \mathit{mul2} \coloneqq a *e +b *d -2*c *f ;\quad \mathit{mul3} \coloneqq a *f +b *e +c *d ;\quad \mathit{RETURN} \! \left(\mathit{mul1} ,\mathit{mul2} ,\mathit{mul3} \right)\quad \boldsymbol{\textrm{end proc}}
\end{dmath}
\mapleinput
{$ \displaystyle \texttt{>\,} \mathit{invert2} \coloneqq \mathrm{proc}(a ,b ,c)\,\mathrm{global}\,\mathit{inv1} ,\mathit{inv2} ,\mathit{inv3} ;\, \mathit{inv1} \coloneqq \frac{(a^{2}+2\cdot b \cdot c)}{\mathit{norm1} (a ,b ,c)};\, \mathit{inv2} \coloneqq \frac{(-a \cdot b -2\cdot c^{2})}{\mathit{norm1} (a ,b ,c)};
\\
 \mathit{inv3} \coloneqq \frac{(b^{2}-a \cdot c)}{\mathit{norm1} (a ,b ,c)};\, \mathit{RETURN} (\mathit{inv1} ,\mathit{inv2} ,\mathit{inv3});\mathrm{end};\, $}

% \mapleresult
\begin{dmath}\label{(6)}
\mathit{invert2} \coloneqq \boldsymbol{\mathrm{proc}}\left(a ,b ,c \right)\quad \boldsymbol{\mathrm{global}}\quad \mathit{inv1} ,\mathit{inv2} ,\mathit{inv3} ;\quad \mathit{inv1} \coloneqq \left(a ^{2}+2*b *c \right)/\mathit{norm1} \! \left(a ,b ,c \right);\quad \mathit{inv2} \coloneqq \left(-b *a -2*c ^{2}\right)/\mathit{norm1} \! \left(a ,b ,c \right);\quad \mathit{inv3} \coloneqq \left(b ^{2}-a *c \right)/\mathit{norm1} \! \left(a ,b ,c \right);\quad \mathit{RETURN} \! \left(\mathit{inv1} ,\mathit{inv2} ,\mathit{inv3} \right)\quad \boldsymbol{\textrm{end proc}}
\end{dmath}
\mapleinput
{$ \displaystyle \texttt{>\,} \mathit{divide3} \coloneqq \mathrm{proc}(a ,b ,c ,d ,e ,f);\mathit{mult2} (a ,b ,c ,\mathit{invert2} (d ,e ,f));\mathrm{end};\, $}

% \mapleresult
\begin{dmath}\label{(7)}
\mathit{divide3} \coloneqq 
\\
\boldsymbol{\mathrm{proc}}\left(a ,b ,c ,d ,e ,f \right)\quad \mathit{mult2} \! \left(a ,b ,c ,\mathit{invert2} \! \left(d ,e ,f \right)\right)\quad \boldsymbol{\textrm{end proc}}
\end{dmath}
\mapleinput
{$ \displaystyle \texttt{>\,}  $}

\begin{Maple Normal}
Divide3 is a procedure to produce the result of dividing two triples of the form (a,b,c) = a +bz + cz
{$ ^{ 2.0} $}If this division produces an element with integer values we say (a,b,c) is divisible.  Next we define our factor base:
\end{Maple Normal}
\begin{Maple Normal}

\end{Maple Normal}
\mapleinput
{$ \displaystyle \texttt{>\,} \,U \coloneqq [1,1,0];A \coloneqq [0,1,0];B \coloneqq [-1,1,0];C \coloneqq [1,0,1];\mathit{D1} \coloneqq [1,1,-1];E \coloneqq [1,-2,0];F \coloneqq [3,0,-1]; $}

% \mapleresult
\begin{dmath*}
U \coloneqq \left[1,1,0\right]
\end{dmath*}
\vspace{-\bigskipamount}
% \mapleresult
\begin{dmath*}
A \coloneqq \left[0,1,0\right]
\end{dmath*}
\vspace{-\bigskipamount}
% \mapleresult
\begin{dmath*}
B \coloneqq \left[-1,1,0\right]
\end{dmath*}
\vspace{-\bigskipamount}
% \mapleresult
\begin{dmath*}
C \coloneqq \left[1,0,1\right]
\end{dmath*}
\vspace{-\bigskipamount}
% \mapleresult
\begin{dmath*}
\mathit{D1} \coloneqq \left[1,1,-1\right]
\end{dmath*}
\vspace{-\bigskipamount}
% \mapleresult
\begin{dmath*}
E \coloneqq \left[1,-2,0\right]
\end{dmath*}
\vspace{-\bigskipamount}
% \mapleresult
\begin{dmath}\label{(8)}
F \coloneqq \left[3,0,-1\right]
\end{dmath}
\begin{Maple Normal}

\end{Maple Normal}
\begin{Maple Normal}

\end{Maple Normal}
\begin{Maple Normal}
If A = [0,1,0], B = [-1,-1,0], C = [1,0,1], D = [1,1,-1], E = [1,-2,0], F = [3,0,-1] then we have prime elements with norm = +/-  2, 3, 5, 11, 17, 23
\end{Maple Normal}
\begin{Maple Normal}

\end{Maple Normal}
\begin{Maple Normal}
We need procedures to decide on divisibility and then how many times we can divide out the prime element. We also need to find how many times we can divide out by a unit element which we choose as U = [1,1,0]
\end{Maple Normal}
\begin{Maple Normal}

\end{Maple Normal}
\begin{Maple Normal}
The 'divisibleby' procedures check if the outcome of the division rule produces integer values in each of the three positions      s[1], s[2], and s[3]. If it does we can perform the 'divideby' procedure to extract all powers of the member of the factor base. 
\end{Maple Normal}
\begin{Maple Normal}

\end{Maple Normal}
\begin{Maple Normal}

\end{Maple Normal}
\mapleinput
{$ \displaystyle \texttt{>\,} \mathit{divisiblebyA} \coloneqq \mathrm{proc}(a ,b ,c);\,s \coloneqq [\mathit{mult2} (a ,b ,c ,\mathit{invert2} (0,1,0))];\,\mathrm{if}\,\mathit{type} (s [1],\mathit{integer})\,\land \,\mathit{type} (s [2],\mathit{integer})\,\land \,\mathit{type} (s [3],\mathit{integer})\,\mathrm{then}\,\mathit{true} \,\mathrm{else}\,\mathit{false} ;\mathrm{fi};\,\mathrm{end};\, \,\mathit{divisiblebyB} \coloneqq \mathrm{proc}(a ,b ,c);\,s \coloneqq [\mathit{mult2} (a ,b ,c ,\mathit{invert2} (-1,1,0))];\,\mathrm{if}\,\mathit{type} (s [1],\mathit{integer})\,\land \,\mathit{type} (s [2],\mathit{integer})\,\land \,\mathit{type} (s [3],\mathit{integer})\,\mathrm{then}\,\mathit{true} \,\mathrm{else}\,\mathit{false} ;\mathrm{fi};\,\mathrm{end};\, \,\mathit{divisiblebyC} \coloneqq \mathrm{proc}(a ,b ,c);\,s \coloneqq [\mathit{mult2} (a ,b ,c ,\mathit{invert2} (1,0,1))];\,\mathrm{if}\,\mathit{type} (s [1],\mathit{integer})\,\land \,\mathit{type} (s [2],\mathit{integer})\,\land \,\mathit{type} (s [3],\mathit{integer})\,\mathrm{then}\,\mathit{true} \,\mathrm{else}\,\mathit{false} ;\mathrm{fi};\,\mathrm{end};\, \,\mathit{divisiblebyD} \coloneqq \mathrm{proc}(a ,b ,c);\,s \coloneqq [\mathit{mult2} (a ,b ,c ,\mathit{invert2} (1,1,-1))];\,\mathrm{if}\,\mathit{type} (s [1],\mathit{integer})\,\land \,\mathit{type} (s [2],\mathit{integer})\,\land \,\mathit{type} (s [3],\mathit{integer})\,\mathrm{then}\,\mathit{true} \,\mathrm{else}\,\mathit{false} ;\mathrm{fi};\,\mathrm{end};\, \mathit{divisiblebyE} \coloneqq \mathrm{proc}(a ,b ,c);\,s \coloneqq [\mathit{mult2} (a ,b ,c ,\mathit{invert2} (1,-2,0))];\,\mathrm{if}\,\mathit{type} (s [1],\mathit{integer})\,\land \,\mathit{type} (s [2],\mathit{integer})\,\land \,\mathit{type} (s [3],\mathit{integer})\,\mathrm{then}\,\mathit{true} \,\mathrm{else}\,\mathit{false} ;\mathrm{fi};\,\mathrm{end};\, \,\mathit{divisiblebyF} \coloneqq \mathrm{proc}(a ,b ,c);\,s \coloneqq [\mathit{mult2} (a ,b ,c ,\mathit{invert2} (3,0,-1))];\,\mathrm{if}\,\mathit{type} (s [1],\mathit{integer})\,\land \,\mathit{type} (s [2],\mathit{integer})\,\land \,\mathit{type} (s [3],\mathit{integer})\,\mathrm{then}\,\mathit{true} \,\mathrm{else}\,\mathit{false} ;\mathrm{fi};\,\mathrm{end};\, \,\mathit{divisiblebyU} \coloneqq \mathrm{proc}(a ,b ,c);\,s \coloneqq [\mathit{mult2} (a ,b ,c ,\mathit{invert2} (1,1,0))];\mathrm{if}\,\mathit{type} (s [1],\mathit{integer})\,\land \,\mathit{type} (s [2],\mathit{integer})\,\land \,\mathit{type} (s [3],\mathit{integer})\,\mathrm{then}\,\mathit{true} \,\mathrm{else}\,\mathit{false} ;\mathrm{fi};\,\mathrm{end};\, 
\\
  $}

% \mapleresult
\href{http://www.maplesoft.com/support/help/errors/view.aspx?path=Warning,%20(in%20divisiblebyA)%20%60s%60%20is%20implicitly%20declared%20local}{Warning, (in divisiblebyA) `s` is implicitly declared local}% \mapleresult
\href{http://www.maplesoft.com/support/help/errors/view.aspx?path=Warning,%20(in%20divisiblebyB)%20%60s%60%20is%20implicitly%20declared%20local}{Warning, (in divisiblebyB) `s` is implicitly declared local}% \mapleresult
\href{http://www.maplesoft.com/support/help/errors/view.aspx?path=Warning,%20(in%20divisiblebyC)%20%60s%60%20is%20implicitly%20declared%20local}{Warning, (in divisiblebyC) `s` is implicitly declared local}% \mapleresult
\href{http://www.maplesoft.com/support/help/errors/view.aspx?path=Warning,%20(in%20divisiblebyD)%20%60s%60%20is%20implicitly%20declared%20local}{Warning, (in divisiblebyD) `s` is implicitly declared local}% \mapleresult
\href{http://www.maplesoft.com/support/help/errors/view.aspx?path=Warning,%20(in%20divisiblebyE)%20%60s%60%20is%20implicitly%20declared%20local}{Warning, (in divisiblebyE) `s` is implicitly declared local}% \mapleresult
\href{http://www.maplesoft.com/support/help/errors/view.aspx?path=Warning,%20(in%20divisiblebyF)%20%60s%60%20is%20implicitly%20declared%20local}{Warning, (in divisiblebyF) `s` is implicitly declared local}% \mapleresult
\href{http://www.maplesoft.com/support/help/errors/view.aspx?path=Warning,%20(in%20divisiblebyU)%20%60s%60%20is%20implicitly%20declared%20local}{Warning, (in divisiblebyU) `s` is implicitly declared local}% \mapleresult
\begin{dmath*}
\mathit{divisiblebyA} \coloneqq 
\\
\boldsymbol{\mathrm{proc}}\left(a ,b ,c \right)\quad \boldsymbol{\mathrm{local}}\quad s ;\quad 
\\
s \coloneqq \left[\mathit{mult2} \! \left(a ,b ,c ,\mathit{invert2} \! \left(0,1,0\right)\right)\right];\quad \boldsymbol{\mathrm{if}}\quad 
\\
\mathit{type} \! \left(s \left[1\right],\mathit{integer} \right)\quad \boldsymbol{\land}\quad \mathit{type} \! \left(s \left[2\right],\mathit{integer} \right)\quad \boldsymbol{\land}\quad 
\\
\mathit{type} \! \left(s \left[3\right],\mathit{integer} \right)\quad 
\\
\boldsymbol{\mathrm{then}}\quad \mathit{true} \quad 
\\
\boldsymbol{\mathrm{else}}\quad \mathit{false} \quad \boldsymbol{\textrm{end if}}\quad 
\\
\boldsymbol{\textrm{end proc}}
\end{dmath*}
\vspace{-\bigskipamount}
% \mapleresult
\begin{dmath*}
\mathit{divisiblebyB} \coloneqq 
\\
\boldsymbol{\mathrm{proc}}\left(a ,b ,c \right)\quad \boldsymbol{\mathrm{local}}\quad s ;\quad 
\\
s \coloneqq \left[\mathit{mult2} \! \left(a ,b ,c ,\mathit{invert2} \! \left(-1,1,0\right)\right)\right];\quad \boldsymbol{\mathrm{if}}\quad 
\\
\mathit{type} \! \left(s \left[1\right],\mathit{integer} \right)\quad \boldsymbol{\land}\quad \mathit{type} \! \left(s \left[2\right],\mathit{integer} \right)\quad \boldsymbol{\land}\quad 
\\
\mathit{type} \! \left(s \left[3\right],\mathit{integer} \right)\quad 
\\
\boldsymbol{\mathrm{then}}\quad \mathit{true} \quad 
\\
\boldsymbol{\mathrm{else}}\quad \mathit{false} \quad \boldsymbol{\textrm{end if}}\quad 
\\
\boldsymbol{\textrm{end proc}}
\end{dmath*}
\vspace{-\bigskipamount}
% \mapleresult
\begin{dmath*}
\mathit{divisiblebyC} \coloneqq 
\\
\boldsymbol{\mathrm{proc}}\left(a ,b ,c \right)\quad \boldsymbol{\mathrm{local}}\quad s ;\quad 
\\
s \coloneqq \left[\mathit{mult2} \! \left(a ,b ,c ,\mathit{invert2} \! \left(1,0,1\right)\right)\right];\quad \boldsymbol{\mathrm{if}}\quad 
\\
\mathit{type} \! \left(s \left[1\right],\mathit{integer} \right)\quad \boldsymbol{\land}\quad \mathit{type} \! \left(s \left[2\right],\mathit{integer} \right)\quad \boldsymbol{\land}\quad 
\\
\mathit{type} \! \left(s \left[3\right],\mathit{integer} \right)\quad 
\\
\boldsymbol{\mathrm{then}}\quad \mathit{true} \quad 
\\
\boldsymbol{\mathrm{else}}\quad \mathit{false} \quad \boldsymbol{\textrm{end if}}\quad 
\\
\boldsymbol{\textrm{end proc}}
\end{dmath*}
\vspace{-\bigskipamount}
% \mapleresult
\begin{dmath*}
\mathit{divisiblebyD} \coloneqq 
\\
\boldsymbol{\mathrm{proc}}\left(a ,b ,c \right)\quad \boldsymbol{\mathrm{local}}\quad s ;\quad 
\\
s \coloneqq \left[\mathit{mult2} \! \left(a ,b ,c ,\mathit{invert2} \! \left(1,1,-1\right)\right)\right];\quad \boldsymbol{\mathrm{if}}\quad 
\\
\mathit{type} \! \left(s \left[1\right],\mathit{integer} \right)\quad \boldsymbol{\land}\quad \mathit{type} \! \left(s \left[2\right],\mathit{integer} \right)\quad \boldsymbol{\land}\quad 
\\
\mathit{type} \! \left(s \left[3\right],\mathit{integer} \right)\quad 
\\
\boldsymbol{\mathrm{then}}\quad \mathit{true} \quad 
\\
\boldsymbol{\mathrm{else}}\quad \mathit{false} \quad \boldsymbol{\textrm{end if}}\quad 
\\
\boldsymbol{\textrm{end proc}}
\end{dmath*}
\vspace{-\bigskipamount}
% \mapleresult
\begin{dmath*}
\mathit{divisiblebyE} \coloneqq 
\\
\boldsymbol{\mathrm{proc}}\left(a ,b ,c \right)\quad \boldsymbol{\mathrm{local}}\quad s ;\quad 
\\
s \coloneqq \left[\mathit{mult2} \! \left(a ,b ,c ,\mathit{invert2} \! \left(1,-2,0\right)\right)\right];\quad \boldsymbol{\mathrm{if}}\quad 
\\
\mathit{type} \! \left(s \left[1\right],\mathit{integer} \right)\quad \boldsymbol{\land}\quad \mathit{type} \! \left(s \left[2\right],\mathit{integer} \right)\quad \boldsymbol{\land}\quad 
\\
\mathit{type} \! \left(s \left[3\right],\mathit{integer} \right)\quad 
\\
\boldsymbol{\mathrm{then}}\quad \mathit{true} \quad 
\\
\boldsymbol{\mathrm{else}}\quad \mathit{false} \quad \boldsymbol{\textrm{end if}}\quad 
\\
\boldsymbol{\textrm{end proc}}
\end{dmath*}
\vspace{-\bigskipamount}
% \mapleresult
\begin{dmath*}
\mathit{divisiblebyF} \coloneqq 
\\
\boldsymbol{\mathrm{proc}}\left(a ,b ,c \right)\quad \boldsymbol{\mathrm{local}}\quad s ;\quad 
\\
s \coloneqq \left[\mathit{mult2} \! \left(a ,b ,c ,\mathit{invert2} \! \left(3,0,-1\right)\right)\right];\quad \boldsymbol{\mathrm{if}}\quad 
\\
\mathit{type} \! \left(s \left[1\right],\mathit{integer} \right)\quad \boldsymbol{\land}\quad \mathit{type} \! \left(s \left[2\right],\mathit{integer} \right)\quad \boldsymbol{\land}\quad 
\\
\mathit{type} \! \left(s \left[3\right],\mathit{integer} \right)\quad 
\\
\boldsymbol{\mathrm{then}}\quad \mathit{true} \quad 
\\
\boldsymbol{\mathrm{else}}\quad \mathit{false} \quad \boldsymbol{\textrm{end if}}\quad 
\\
\boldsymbol{\textrm{end proc}}
\end{dmath*}
\vspace{-\bigskipamount}
% \mapleresult
\begin{dmath}\label{(9)}
\mathit{divisiblebyU} \coloneqq 
\\
\boldsymbol{\mathrm{proc}}\left(a ,b ,c \right)\quad \boldsymbol{\mathrm{local}}\quad s ;\quad 
\\
s \coloneqq \left[\mathit{mult2} \! \left(a ,b ,c ,\mathit{invert2} \! \left(1,1,0\right)\right)\right];\quad \boldsymbol{\mathrm{if}}\quad 
\\
\mathit{type} \! \left(s \left[1\right],\mathit{integer} \right)\quad \boldsymbol{\land}\quad \mathit{type} \! \left(s \left[2\right],\mathit{integer} \right)\quad \boldsymbol{\land}\quad 
\\
\mathit{type} \! \left(s \left[3\right],\mathit{integer} \right)\quad 
\\
\boldsymbol{\mathrm{then}}\quad \mathit{true} \quad 
\\
\boldsymbol{\mathrm{else}}\quad \mathit{false} \quad \boldsymbol{\textrm{end if}}\quad 
\\
\boldsymbol{\textrm{end proc}}
\end{dmath}
\mapleinput
{$ \displaystyle \texttt{>\,} \,\mathit{dividebyA} \coloneqq \mathrm{proc}(a ,b ,c)\,\mathrm{global}\,\mathit{countA} ,s ;t ;\,\mathit{oldk1} \coloneqq a \colon \mathit{oldk2} \coloneqq b \colon \mathit{oldk3} \coloneqq c \colon \mathit{countA} \coloneqq 0;\,s \coloneqq [a ,b ,c];\mathrm{while}\,\mathit{divisiblebyA} (s [1],s [2],s [3])=\,\mathit{true} \,\mathrm{do}\,\mathit{countA} \coloneqq \mathit{countA} +1;\,\mathit{k1} \coloneqq s [1];\mathit{k2} \coloneqq s [2];\mathit{k3} \coloneqq s [3];s \coloneqq [\mathit{divide3} (\mathit{k1} ,\mathit{k2} ,\mathit{k3} ,0,1,0)];\mathrm{od};\,s ;t \coloneqq (s [1],s [2],s [3]);\mathrm{end}; $}

% \mapleresult
\href{http://www.maplesoft.com/support/help/errors/view.aspx?path=Warning,%20(in%20dividebyA)%20%60oldk1%60%20is%20implicitly%20declared%20local}{Warning, (in dividebyA) `oldk1` is implicitly declared local}% \mapleresult
\href{http://www.maplesoft.com/support/help/errors/view.aspx?path=Warning,%20(in%20dividebyA)%20%60oldk2%60%20is%20implicitly%20declared%20local}{Warning, (in dividebyA) `oldk2` is implicitly declared local}% \mapleresult
\href{http://www.maplesoft.com/support/help/errors/view.aspx?path=Warning,%20(in%20dividebyA)%20%60oldk3%60%20is%20implicitly%20declared%20local}{Warning, (in dividebyA) `oldk3` is implicitly declared local}% \mapleresult
\href{http://www.maplesoft.com/support/help/errors/view.aspx?path=Warning,%20(in%20dividebyA)%20%60k1%60%20is%20implicitly%20declared%20local}{Warning, (in dividebyA) `k1` is implicitly declared local}% \mapleresult
\href{http://www.maplesoft.com/support/help/errors/view.aspx?path=Warning,%20(in%20dividebyA)%20%60k2%60%20is%20implicitly%20declared%20local}{Warning, (in dividebyA) `k2` is implicitly declared local}% \mapleresult
\href{http://www.maplesoft.com/support/help/errors/view.aspx?path=Warning,%20(in%20dividebyA)%20%60k3%60%20is%20implicitly%20declared%20local}{Warning, (in dividebyA) `k3` is implicitly declared local}% \mapleresult
\href{http://www.maplesoft.com/support/help/errors/view.aspx?path=Warning,%20(in%20dividebyA)%20%60t%60%20is%20implicitly%20declared%20local}{Warning, (in dividebyA) `t` is implicitly declared local}% \mapleresult
\begin{dmath}\label{(10)}
\mathit{dividebyA} \coloneqq \boldsymbol{\mathrm{proc}}\left(a ,b ,c \right)\quad \boldsymbol{\mathrm{local}}\quad \mathit{oldk1} ,\mathit{oldk2} ,\mathit{oldk3} ,\mathit{k1} ,\mathit{k2} ,\mathit{k3} ,t ;\quad \boldsymbol{\mathrm{global}}\quad \mathit{countA} ,s ;\quad t ;\quad \mathit{oldk1} \coloneqq a ;\quad \mathit{oldk2} \coloneqq b ;\quad \mathit{oldk3} \coloneqq c ;\quad \mathit{countA} \coloneqq 0;\quad s \coloneqq \left[a ,b ,c \right];\quad \boldsymbol{\mathrm{while}}\quad \mathit{divisiblebyA} \! \left(s \left[1\right],s \left[2\right],s \left[3\right]\right)=\mathit{true} \quad \boldsymbol{\mathrm{do}}\quad \mathit{countA} \coloneqq \mathit{countA} +1;\quad \mathit{k1} \coloneqq s \left[1\right];\quad \mathit{k2} \coloneqq s \left[2\right];\quad \mathit{k3} \coloneqq s \left[3\right];\quad s \coloneqq \left[\mathit{divide3} \! \left(\mathit{k1} ,\mathit{k2} ,\mathit{k3} ,0,1,0\right)\right]\quad \boldsymbol{\textrm{end do}};\quad s ;\quad t \coloneqq s \left[1\right],s \left[2\right],s \left[3\right]\quad \boldsymbol{\textrm{end proc}}
\end{dmath}
\mapleinput
{$ \displaystyle \texttt{>\,} \,\mathit{dividebyB} \coloneqq \mathrm{proc}(a ,b ,c)\,\mathrm{global}\,\mathit{countB} ,s ;t ;\,\mathit{oldk1} \coloneqq a \colon \mathit{oldk2} \coloneqq b \colon \mathit{oldk3} \coloneqq c \colon \mathit{countB} \coloneqq 0;\,s \coloneqq [a ,b ,c];;\mathrm{while}\,\mathit{divisiblebyB} (s [1],s [2],s [3])\,\mathrm{do}\,\mathit{countB} \coloneqq \mathit{countB} +1;\,\mathit{k1} \coloneqq s [1];\mathit{k2} \coloneqq s [2];\mathit{k3} \coloneqq s [3];s \coloneqq [\mathit{divide3} (\mathit{k1} ,\mathit{k2} ,\mathit{k3} ,-1,1,0)];\mathrm{od};\,s ;t \coloneqq (s [1],s [2],s [3]);\mathrm{end}; $}

% \mapleresult
\href{http://www.maplesoft.com/support/help/errors/view.aspx?path=Warning,%20(in%20dividebyB)%20%60oldk1%60%20is%20implicitly%20declared%20local}{Warning, (in dividebyB) `oldk1` is implicitly declared local}% \mapleresult
\href{http://www.maplesoft.com/support/help/errors/view.aspx?path=Warning,%20(in%20dividebyB)%20%60oldk2%60%20is%20implicitly%20declared%20local}{Warning, (in dividebyB) `oldk2` is implicitly declared local}% \mapleresult
\href{http://www.maplesoft.com/support/help/errors/view.aspx?path=Warning,%20(in%20dividebyB)%20%60oldk3%60%20is%20implicitly%20declared%20local}{Warning, (in dividebyB) `oldk3` is implicitly declared local}% \mapleresult
\href{http://www.maplesoft.com/support/help/errors/view.aspx?path=Warning,%20(in%20dividebyB)%20%60k1%60%20is%20implicitly%20declared%20local}{Warning, (in dividebyB) `k1` is implicitly declared local}% \mapleresult
\href{http://www.maplesoft.com/support/help/errors/view.aspx?path=Warning,%20(in%20dividebyB)%20%60k2%60%20is%20implicitly%20declared%20local}{Warning, (in dividebyB) `k2` is implicitly declared local}% \mapleresult
\href{http://www.maplesoft.com/support/help/errors/view.aspx?path=Warning,%20(in%20dividebyB)%20%60k3%60%20is%20implicitly%20declared%20local}{Warning, (in dividebyB) `k3` is implicitly declared local}% \mapleresult
\href{http://www.maplesoft.com/support/help/errors/view.aspx?path=Warning,%20(in%20dividebyB)%20%60t%60%20is%20implicitly%20declared%20local}{Warning, (in dividebyB) `t` is implicitly declared local}% \mapleresult
\begin{dmath}\label{(11)}
\mathit{dividebyB} \coloneqq \boldsymbol{\mathrm{proc}}\left(a ,b ,c \right)\quad \boldsymbol{\mathrm{local}}\quad \mathit{oldk1} ,\mathit{oldk2} ,\mathit{oldk3} ,\mathit{k1} ,\mathit{k2} ,\mathit{k3} ,t ;\quad \boldsymbol{\mathrm{global}}\quad \mathit{countB} ,s ;\quad t ;\quad \mathit{oldk1} \coloneqq a ;\quad \mathit{oldk2} \coloneqq b ;\quad \mathit{oldk3} \coloneqq c ;\quad \mathit{countB} \coloneqq 0;\quad s \coloneqq \left[a ,b ,c \right];\quad \boldsymbol{\mathrm{while}}\quad \mathit{divisiblebyB} \! \left(s \left[1\right],s \left[2\right],s \left[3\right]\right)\quad \boldsymbol{\mathrm{do}}\quad \mathit{countB} \coloneqq \mathit{countB} +1;\quad \mathit{k1} \coloneqq s \left[1\right];\quad \mathit{k2} \coloneqq s \left[2\right];\quad \mathit{k3} \coloneqq s \left[3\right];\quad s \coloneqq \left[\mathit{divide3} \! \left(\mathit{k1} ,\mathit{k2} ,\mathit{k3} ,-1,1,0\right)\right]\quad \boldsymbol{\textrm{end do}};\quad s ;\quad t \coloneqq s \left[1\right],s \left[2\right],s \left[3\right]\quad \boldsymbol{\textrm{end proc}}
\end{dmath}
\mapleinput
{$ \displaystyle \texttt{>\,} \,\mathit{dividebyC} \coloneqq \mathrm{proc}(a ,b ,c)\,\mathrm{global}\,\mathit{countC} ,s ,t ;\,\mathit{oldk1} \coloneqq a \colon \mathit{oldk2} \coloneqq b \colon \mathit{oldk3} \coloneqq c \colon \mathit{countC} \coloneqq 0;\,s \coloneqq [a ,b ,c];;\mathrm{while}\,\mathit{divisiblebyC} (s [1],s [2],s [3])\,\mathrm{do}\,\mathit{countC} \coloneqq \mathit{countC} +1;\,\mathit{k1} \coloneqq s [1];\mathit{k2} \coloneqq s [2];\mathit{k3} \coloneqq s [3];s \coloneqq [\mathit{divide3} (\mathit{k1} ,\mathit{k2} ,\mathit{k3} ,1,0,1)];\mathrm{od};s ;t \coloneqq (s [1],s [2],s [3]);\mathrm{end}; $}

% \mapleresult
\href{http://www.maplesoft.com/support/help/errors/view.aspx?path=Warning,%20(in%20dividebyC)%20%60oldk1%60%20is%20implicitly%20declared%20local}{Warning, (in dividebyC) `oldk1` is implicitly declared local}% \mapleresult
\href{http://www.maplesoft.com/support/help/errors/view.aspx?path=Warning,%20(in%20dividebyC)%20%60oldk2%60%20is%20implicitly%20declared%20local}{Warning, (in dividebyC) `oldk2` is implicitly declared local}% \mapleresult
\href{http://www.maplesoft.com/support/help/errors/view.aspx?path=Warning,%20(in%20dividebyC)%20%60oldk3%60%20is%20implicitly%20declared%20local}{Warning, (in dividebyC) `oldk3` is implicitly declared local}% \mapleresult
\href{http://www.maplesoft.com/support/help/errors/view.aspx?path=Warning,%20(in%20dividebyC)%20%60k1%60%20is%20implicitly%20declared%20local}{Warning, (in dividebyC) `k1` is implicitly declared local}% \mapleresult
\href{http://www.maplesoft.com/support/help/errors/view.aspx?path=Warning,%20(in%20dividebyC)%20%60k2%60%20is%20implicitly%20declared%20local}{Warning, (in dividebyC) `k2` is implicitly declared local}% \mapleresult
\href{http://www.maplesoft.com/support/help/errors/view.aspx?path=Warning,%20(in%20dividebyC)%20%60k3%60%20is%20implicitly%20declared%20local}{Warning, (in dividebyC) `k3` is implicitly declared local}% \mapleresult
\begin{dmath}\label{(12)}
\mathit{dividebyC} \coloneqq \boldsymbol{\mathrm{proc}}\left(a ,b ,c \right)\quad \boldsymbol{\mathrm{local}}\quad \mathit{oldk1} ,\mathit{oldk2} ,\mathit{oldk3} ,\mathit{k1} ,\mathit{k2} ,\mathit{k3} ;\quad \boldsymbol{\mathrm{global}}\quad \mathit{countC} ,s ,t ;\quad \mathit{oldk1} \coloneqq a ;\quad \mathit{oldk2} \coloneqq b ;\quad \mathit{oldk3} \coloneqq c ;\quad \mathit{countC} \coloneqq 0;\quad s \coloneqq \left[a ,b ,c \right];\quad \boldsymbol{\mathrm{while}}\quad \mathit{divisiblebyC} \! \left(s \left[1\right],s \left[2\right],s \left[3\right]\right)\quad \boldsymbol{\mathrm{do}}\quad \mathit{countC} \coloneqq \mathit{countC} +1;\quad \mathit{k1} \coloneqq s \left[1\right];\quad \mathit{k2} \coloneqq s \left[2\right];\quad \mathit{k3} \coloneqq s \left[3\right];\quad s \coloneqq \left[\mathit{divide3} \! \left(\mathit{k1} ,\mathit{k2} ,\mathit{k3} ,1,0,1\right)\right]\quad \boldsymbol{\textrm{end do}};\quad s ;\quad t \coloneqq s \left[1\right],s \left[2\right],s \left[3\right]\quad \boldsymbol{\textrm{end proc}}
\end{dmath}
\mapleinput
{$ \displaystyle \texttt{>\,} \,\mathit{dividebyD} \coloneqq \mathrm{proc}(a ,b ,c)\,\mathrm{global}\,\mathit{countD} ,s ,t ;\,\mathit{oldk1} \coloneqq a \colon \mathit{oldk2} \coloneqq b \colon \mathit{oldk3} \coloneqq c \colon \mathit{countD} \coloneqq 0;\,s \coloneqq [a ,b ,c];;\mathrm{while}\,\mathit{divisiblebyD} (s [1],s [2],s [3])\,\mathrm{do}\,\mathit{countD} \coloneqq \mathit{countD} +1;\,\mathit{k1} \coloneqq s [1];\mathit{k2} \coloneqq s [2];\mathit{k3} \coloneqq s [3];s \coloneqq [\mathit{divide3} (\mathit{k1} ,\mathit{k2} ,\mathit{k3} ,1,1,-1)];\mathrm{od};\,s ;t \coloneqq (s [1],s [2],s [3]);\mathrm{end}; $}

% \mapleresult
\href{http://www.maplesoft.com/support/help/errors/view.aspx?path=Warning,%20(in%20dividebyD)%20%60oldk1%60%20is%20implicitly%20declared%20local}{Warning, (in dividebyD) `oldk1` is implicitly declared local}% \mapleresult
\href{http://www.maplesoft.com/support/help/errors/view.aspx?path=Warning,%20(in%20dividebyD)%20%60oldk2%60%20is%20implicitly%20declared%20local}{Warning, (in dividebyD) `oldk2` is implicitly declared local}% \mapleresult
\href{http://www.maplesoft.com/support/help/errors/view.aspx?path=Warning,%20(in%20dividebyD)%20%60oldk3%60%20is%20implicitly%20declared%20local}{Warning, (in dividebyD) `oldk3` is implicitly declared local}% \mapleresult
\href{http://www.maplesoft.com/support/help/errors/view.aspx?path=Warning,%20(in%20dividebyD)%20%60k1%60%20is%20implicitly%20declared%20local}{Warning, (in dividebyD) `k1` is implicitly declared local}% \mapleresult
\href{http://www.maplesoft.com/support/help/errors/view.aspx?path=Warning,%20(in%20dividebyD)%20%60k2%60%20is%20implicitly%20declared%20local}{Warning, (in dividebyD) `k2` is implicitly declared local}% \mapleresult
\href{http://www.maplesoft.com/support/help/errors/view.aspx?path=Warning,%20(in%20dividebyD)%20%60k3%60%20is%20implicitly%20declared%20local}{Warning, (in dividebyD) `k3` is implicitly declared local}% \mapleresult
\begin{dmath}\label{(13)}
\mathit{dividebyD} \coloneqq \boldsymbol{\mathrm{proc}}\left(a ,b ,c \right)\quad \boldsymbol{\mathrm{local}}\quad \mathit{oldk1} ,\mathit{oldk2} ,\mathit{oldk3} ,\mathit{k1} ,\mathit{k2} ,\mathit{k3} ;\quad \boldsymbol{\mathrm{global}}\quad \mathit{countD} ,s ,t ;\quad \mathit{oldk1} \coloneqq a ;\quad \mathit{oldk2} \coloneqq b ;\quad \mathit{oldk3} \coloneqq c ;\quad \mathit{countD} \coloneqq 0;\quad s \coloneqq \left[a ,b ,c \right];\quad \boldsymbol{\mathrm{while}}\quad \mathit{divisiblebyD} \! \left(s \left[1\right],s \left[2\right],s \left[3\right]\right)\quad \boldsymbol{\mathrm{do}}\quad \mathit{countD} \coloneqq \mathit{countD} +1;\quad \mathit{k1} \coloneqq s \left[1\right];\quad \mathit{k2} \coloneqq s \left[2\right];\quad \mathit{k3} \coloneqq s \left[3\right];\quad s \coloneqq \left[\mathit{divide3} \! \left(\mathit{k1} ,\mathit{k2} ,\mathit{k3} ,1,1,-1\right)\right]\quad \boldsymbol{\textrm{end do}};\quad s ;\quad t \coloneqq s \left[1\right],s \left[2\right],s \left[3\right]\quad \boldsymbol{\textrm{end proc}}
\end{dmath}
\mapleinput
{$ \displaystyle \texttt{>\,} \,\mathit{dividebyE} \coloneqq \mathrm{proc}(a ,b ,c)\,\mathrm{global}\,\mathit{countE} ,s ,t ;\,\mathit{oldk1} \coloneqq a \colon \mathit{oldk2} \coloneqq b \colon \mathit{oldk3} \coloneqq c \colon \mathit{countE} \coloneqq 0;\,s \coloneqq [a ,b ,c];\mathrm{while}\,\mathit{divisiblebyE} (s [1],s [2],s [3])\,\mathrm{do}\,\mathit{countE} \coloneqq \mathit{countE} +1;\,\mathit{k1} \coloneqq s [1];\mathit{k2} \coloneqq s [2];\mathit{k3} \coloneqq s [3];s \coloneqq [\mathit{divide3} (\mathit{k1} ,\mathit{k2} ,\mathit{k3} ,1,-2,0)];\mathrm{od};\,s ;t \coloneqq (s [1],s [2],s [3]);\mathrm{end}; $}

% \mapleresult
\href{http://www.maplesoft.com/support/help/errors/view.aspx?path=Warning,%20(in%20dividebyE)%20%60oldk1%60%20is%20implicitly%20declared%20local}{Warning, (in dividebyE) `oldk1` is implicitly declared local}% \mapleresult
\href{http://www.maplesoft.com/support/help/errors/view.aspx?path=Warning,%20(in%20dividebyE)%20%60oldk2%60%20is%20implicitly%20declared%20local}{Warning, (in dividebyE) `oldk2` is implicitly declared local}% \mapleresult
\href{http://www.maplesoft.com/support/help/errors/view.aspx?path=Warning,%20(in%20dividebyE)%20%60oldk3%60%20is%20implicitly%20declared%20local}{Warning, (in dividebyE) `oldk3` is implicitly declared local}% \mapleresult
\href{http://www.maplesoft.com/support/help/errors/view.aspx?path=Warning,%20(in%20dividebyE)%20%60k1%60%20is%20implicitly%20declared%20local}{Warning, (in dividebyE) `k1` is implicitly declared local}% \mapleresult
\href{http://www.maplesoft.com/support/help/errors/view.aspx?path=Warning,%20(in%20dividebyE)%20%60k2%60%20is%20implicitly%20declared%20local}{Warning, (in dividebyE) `k2` is implicitly declared local}% \mapleresult
\href{http://www.maplesoft.com/support/help/errors/view.aspx?path=Warning,%20(in%20dividebyE)%20%60k3%60%20is%20implicitly%20declared%20local}{Warning, (in dividebyE) `k3` is implicitly declared local}% \mapleresult
\begin{dmath}\label{(14)}
\mathit{dividebyE} \coloneqq \boldsymbol{\mathrm{proc}}\left(a ,b ,c \right)\quad \boldsymbol{\mathrm{local}}\quad \mathit{oldk1} ,\mathit{oldk2} ,\mathit{oldk3} ,\mathit{k1} ,\mathit{k2} ,\mathit{k3} ;\quad \boldsymbol{\mathrm{global}}\quad \mathit{countE} ,s ,t ;\quad \mathit{oldk1} \coloneqq a ;\quad \mathit{oldk2} \coloneqq b ;\quad \mathit{oldk3} \coloneqq c ;\quad \mathit{countE} \coloneqq 0;\quad s \coloneqq \left[a ,b ,c \right];\quad \boldsymbol{\mathrm{while}}\quad \mathit{divisiblebyE} \! \left(s \left[1\right],s \left[2\right],s \left[3\right]\right)\quad \boldsymbol{\mathrm{do}}\quad \mathit{countE} \coloneqq \mathit{countE} +1;\quad \mathit{k1} \coloneqq s \left[1\right];\quad \mathit{k2} \coloneqq s \left[2\right];\quad \mathit{k3} \coloneqq s \left[3\right];\quad s \coloneqq \left[\mathit{divide3} \! \left(\mathit{k1} ,\mathit{k2} ,\mathit{k3} ,1,-2,0\right)\right]\quad \boldsymbol{\textrm{end do}};\quad s ;\quad t \coloneqq s \left[1\right],s \left[2\right],s \left[3\right]\quad \boldsymbol{\textrm{end proc}}
\end{dmath}
\mapleinput
{$ \displaystyle \texttt{>\,} \mathit{dividebyF} \coloneqq \mathrm{proc}(a ,b ,c)\,\mathrm{global}\,\mathit{countF} ,s ,t ;\,\mathit{oldk1} \coloneqq a \colon \mathit{oldk2} \coloneqq b \colon \mathit{oldk3} \coloneqq c \colon \mathit{countF} \coloneqq 0;\,s \coloneqq [a ,b ,c];;\mathrm{while}\,\mathit{divisiblebyF} (s [1],s [2],s [3])\,\mathrm{do}\,\mathit{countF} \coloneqq \mathit{countF} +1;\,\mathit{k1} \coloneqq s [1];\mathit{k2} \coloneqq s [2];\mathit{k3} \coloneqq s [3];s \coloneqq [\mathit{divide3} (\mathit{k1} ,\mathit{k2} ,\mathit{k3} ,3,0,-1)];\mathrm{od};\,s ;t \coloneqq (s [1],s [2],s [3]);\mathrm{end}; $}

% \mapleresult
\href{http://www.maplesoft.com/support/help/errors/view.aspx?path=Warning,%20(in%20dividebyF)%20%60oldk1%60%20is%20implicitly%20declared%20local}{Warning, (in dividebyF) `oldk1` is implicitly declared local}% \mapleresult
\href{http://www.maplesoft.com/support/help/errors/view.aspx?path=Warning,%20(in%20dividebyF)%20%60oldk2%60%20is%20implicitly%20declared%20local}{Warning, (in dividebyF) `oldk2` is implicitly declared local}% \mapleresult
\href{http://www.maplesoft.com/support/help/errors/view.aspx?path=Warning,%20(in%20dividebyF)%20%60oldk3%60%20is%20implicitly%20declared%20local}{Warning, (in dividebyF) `oldk3` is implicitly declared local}% \mapleresult
\href{http://www.maplesoft.com/support/help/errors/view.aspx?path=Warning,%20(in%20dividebyF)%20%60k1%60%20is%20implicitly%20declared%20local}{Warning, (in dividebyF) `k1` is implicitly declared local}% \mapleresult
\href{http://www.maplesoft.com/support/help/errors/view.aspx?path=Warning,%20(in%20dividebyF)%20%60k2%60%20is%20implicitly%20declared%20local}{Warning, (in dividebyF) `k2` is implicitly declared local}% \mapleresult
\href{http://www.maplesoft.com/support/help/errors/view.aspx?path=Warning,%20(in%20dividebyF)%20%60k3%60%20is%20implicitly%20declared%20local}{Warning, (in dividebyF) `k3` is implicitly declared local}% \mapleresult
\begin{dmath}\label{(15)}
\mathit{dividebyF} \coloneqq \boldsymbol{\mathrm{proc}}\left(a ,b ,c \right)\quad \boldsymbol{\mathrm{local}}\quad \mathit{oldk1} ,\mathit{oldk2} ,\mathit{oldk3} ,\mathit{k1} ,\mathit{k2} ,\mathit{k3} ;\quad \boldsymbol{\mathrm{global}}\quad \mathit{countF} ,s ,t ;\quad \mathit{oldk1} \coloneqq a ;\quad \mathit{oldk2} \coloneqq b ;\quad \mathit{oldk3} \coloneqq c ;\quad \mathit{countF} \coloneqq 0;\quad s \coloneqq \left[a ,b ,c \right];\quad \boldsymbol{\mathrm{while}}\quad \mathit{divisiblebyF} \! \left(s \left[1\right],s \left[2\right],s \left[3\right]\right)\quad \boldsymbol{\mathrm{do}}\quad \mathit{countF} \coloneqq \mathit{countF} +1;\quad \mathit{k1} \coloneqq s \left[1\right];\quad \mathit{k2} \coloneqq s \left[2\right];\quad \mathit{k3} \coloneqq s \left[3\right];\quad s \coloneqq \left[\mathit{divide3} \! \left(\mathit{k1} ,\mathit{k2} ,\mathit{k3} ,3,0,-1\right)\right]\quad \boldsymbol{\textrm{end do}};\quad s ;\quad t \coloneqq s \left[1\right],s \left[2\right],s \left[3\right]\quad \boldsymbol{\textrm{end proc}}
\end{dmath}
\mapleinput
{$ \displaystyle \texttt{>\,}  $}

\mapleinput
{$ \displaystyle \texttt{>\,} \mathit{dividebyU2} \coloneqq \mathrm{proc}(a ,b ,c)\,\mathrm{global}\,\mathit{countU} ,s ,\mathit{sign1} ;\,\mathit{sign1} \coloneqq -1;\,\mathrm{if}\,\mathit{norm1} (a ,b ,c)\neq \,1\,\land \,\mathit{norm1} (a ,b ,c)\neq -1\,\mathrm{then}\,\mathit{false} \,\mathrm{else}\,\mathit{oldk1} \coloneqq a \colon \mathit{oldk2} \coloneqq b \colon \mathit{oldk3} \coloneqq c ;\,s \coloneqq [a ,b ,c];\,\mathrm{if}\,s =[0,0,0]\,\mathrm{then}\,\mathit{false} \,\mathrm{elif}\,s =[-1,1,-1]\,\mathrm{then}\,\mathit{false} \,\mathrm{elif}\,s =[1,-1,1]\,\mathrm{then}\,\mathit{false} \,\mathrm{elif}\,s =[1,0,0]\,\mathrm{then}\,\mathit{countU} \coloneqq 0;\mathit{sign1} \coloneqq 0;\,\mathit{print} (\mathit{countU});\mathit{print} (\mathit{sign1});\mathrm{elif}\,s =\,[-1,0,0]\,\mathrm{then}\,\mathit{countU} \coloneqq 0;\mathit{sign1} \coloneqq 1;\,\mathit{print} (\mathit{countU});\mathit{print} (\mathit{sign1});\,\mathrm{else}\,\mathrm{if}\,s =[-1,1,-1]\,\mathrm{then}\,\mathit{false} \,\mathrm{else}\,\mathit{countU} \coloneqq 1;\mathrm{while}\,\neg ((s [1]=1\,\land \,s [2]=1\,\land \,s [3]=0)\,\lor \,(s [1]=-1\,\land \,s [2]=-1\,\land \,s [3]=0))\,\mathrm{do}\,\mathit{countU} \coloneqq \mathit{countU} +1;\,\mathit{k1} \coloneqq s [1];\mathit{k2} \coloneqq s [2];\mathit{k3} \coloneqq s [3];s \coloneqq [\mathit{divide3} (\mathit{k1} ,\mathit{k2} ,\mathit{k3} ,1,1,0)];\mathrm{od};\mathrm{if}\,\mathit{k1} =1\,\mathrm{then}\,\mathit{sign1} \coloneqq 1\,\mathrm{else}\,\mathit{sign1} \coloneqq 1\,\mathrm{fi};\,\mathrm{fi};\mathrm{fi};\mathrm{fi};\mathrm{end}; $}

% \mapleresult
\href{http://www.maplesoft.com/support/help/errors/view.aspx?path=Warning,%20(in%20dividebyU2)%20%60oldk1%60%20is%20implicitly%20declared%20local}{Warning, (in dividebyU2) `oldk1` is implicitly declared local}% \mapleresult
\href{http://www.maplesoft.com/support/help/errors/view.aspx?path=Warning,%20(in%20dividebyU2)%20%60oldk2%60%20is%20implicitly%20declared%20local}{Warning, (in dividebyU2) `oldk2` is implicitly declared local}% \mapleresult
\href{http://www.maplesoft.com/support/help/errors/view.aspx?path=Warning,%20(in%20dividebyU2)%20%60oldk3%60%20is%20implicitly%20declared%20local}{Warning, (in dividebyU2) `oldk3` is implicitly declared local}% \mapleresult
\href{http://www.maplesoft.com/support/help/errors/view.aspx?path=Warning,%20(in%20dividebyU2)%20%60k1%60%20is%20implicitly%20declared%20local}{Warning, (in dividebyU2) `k1` is implicitly declared local}% \mapleresult
\href{http://www.maplesoft.com/support/help/errors/view.aspx?path=Warning,%20(in%20dividebyU2)%20%60k2%60%20is%20implicitly%20declared%20local}{Warning, (in dividebyU2) `k2` is implicitly declared local}% \mapleresult
\href{http://www.maplesoft.com/support/help/errors/view.aspx?path=Warning,%20(in%20dividebyU2)%20%60k3%60%20is%20implicitly%20declared%20local}{Warning, (in dividebyU2) `k3` is implicitly declared local}% \mapleresult
\begin{dmath}\label{(16)}
\mathit{dividebyU2} \coloneqq \boldsymbol{\mathrm{proc}}\left(a ,b ,c \right)\quad \boldsymbol{\mathrm{local}}\quad \mathit{oldk1} ,\mathit{oldk2} ,\mathit{oldk3} ,\mathit{k1} ,\mathit{k2} ,\mathit{k3} ;\quad \boldsymbol{\mathrm{global}}\quad \mathit{countU} ,s ,\mathit{sign1} ;\quad \mathit{sign1} \coloneqq -1;\quad \boldsymbol{\mathrm{if}}\quad \mathit{norm1} \! \left(a ,b ,c \right)<>1\quad \boldsymbol{\land}\quad \mathit{norm1} \! \left(a ,b ,c \right)<>-1\quad 
\\
\boldsymbol{\mathrm{then}}\quad \mathit{false} \quad \boldsymbol{\mathrm{else}}\quad \mathit{oldk1} \coloneqq a ;\quad \mathit{oldk2} \coloneqq b ;\quad \mathit{oldk3} \coloneqq c ;\quad s \coloneqq \left[a ,b ,c \right];\quad \boldsymbol{\mathrm{if}}\quad s =\left[0,0,0\right]\quad \boldsymbol{\mathrm{then}}\quad \mathit{false} \quad \boldsymbol{\mathrm{elif}}\quad s =\left[-1,1,-1\right]\quad \boldsymbol{\mathrm{then}}\quad \mathit{false} \quad \boldsymbol{\mathrm{elif}}\quad s =\left[1,-1,1\right]\quad \boldsymbol{\mathrm{then}}\quad \mathit{false} \quad \boldsymbol{\mathrm{elif}}\quad s =\left[1,0,0\right]\quad \boldsymbol{\mathrm{then}}\quad 
\\
\mathit{countU} \coloneqq 0;\quad \mathit{sign1} \coloneqq 0;\quad \mathit{print} \! \left(\mathit{countU} \right);\quad \mathit{print} \! \left(\mathit{sign1} \right)\quad \boldsymbol{\mathrm{elif}}\quad s =\left[-1,0,0\right]\quad \boldsymbol{\mathrm{then}}\quad 
\\
\mathit{countU} \coloneqq 0;\quad \mathit{sign1} \coloneqq 1;\quad \mathit{print} \! \left(\mathit{countU} \right);\quad \mathit{print} \! \left(\mathit{sign1} \right)\quad \boldsymbol{\mathrm{else}}\quad \boldsymbol{\mathrm{if}}\quad s =\left[-1,1,-1\right]\quad \boldsymbol{\mathrm{then}}\quad \mathit{false} \quad \boldsymbol{\mathrm{else}}\quad \mathit{countU} \coloneqq 1;\quad \boldsymbol{\mathrm{while}}\quad 
\\
\mathit{not} \! \left(s \left[1\right]=1\quad \boldsymbol{\land}\quad s \left[2\right]=1\quad \boldsymbol{\land}\quad s \left[3\right]=0\quad \boldsymbol{\lor}\quad 
\\
s \left[1\right]=-1\quad \boldsymbol{\land}\quad s \left[2\right]=-1\quad \boldsymbol{\land}\quad s \left[3\right]=0\right)\quad 
\\
\boldsymbol{\mathrm{do}}\quad \mathit{countU} \coloneqq \mathit{countU} +1;\quad \mathit{k1} \coloneqq s \left[1\right];\quad \mathit{k2} \coloneqq s \left[2\right];\quad \mathit{k3} \coloneqq s \left[3\right];\quad s \coloneqq \left[\mathit{divide3} \! \left(\mathit{k1} ,\mathit{k2} ,\mathit{k3} ,1,1,0\right)\right]\quad \boldsymbol{\textrm{end do}};\quad \boldsymbol{\mathrm{if}}\quad \mathit{k1} =1\quad \boldsymbol{\mathrm{then}}\quad \mathit{sign1} \coloneqq 1\quad \boldsymbol{\mathrm{else}}\quad \mathit{sign1} \coloneqq 1\quad \boldsymbol{\textrm{end if}}\quad \boldsymbol{\textrm{end if}}\quad \boldsymbol{\textrm{end if}}\quad \boldsymbol{\textrm{end if}}\quad \boldsymbol{\textrm{end proc}}
\end{dmath}
\mapleinput
{$ \displaystyle \texttt{>\,}  $}

\begin{Maple Normal}
The factorisation procedure below performs the divisions by each of the factor bases
\end{Maple Normal}
\begin{Maple Normal}

\end{Maple Normal}
\mapleinput
{$ \displaystyle \texttt{>\,} \mathit{factorisation} \coloneqq \mathrm{proc}(a ,b ,c);\,\mathrm{if}\,a =0\,\land \,b =0\,\land \,c =0\,\mathrm{then}\,\mathit{false} \,\mathrm{else}\,\mathit{dividebyU2} (\mathit{dividebyA} (\mathit{dividebyB} (\mathit{dividebyC} (\mathit{dividebyD} (\mathit{dividebyE} (\mathit{dividebyF} (a ,b ,c)))))))\colon \,\mathrm{if}\,\mathit{sign1} \ge 0\,\mathrm{then}\,\mathit{print} (a ,b ,[\mathit{sign1} ,\mathit{countU} ,\mathit{countA} ,\mathit{countB} ,\mathit{countC} ,\mathit{countD} ,\mathit{countE} ,\mathit{countF}])\mathrm{else}\,\mathit{print} (a ,b ,\mathit{False\,-\,does\,not\,factor\,});\mathit{false} ;\mathrm{fi};\mathrm{fi};\mathrm{end};\, $}

% \mapleresult
\begin{dmath}\label{(17)}
\mathit{factorisation} \coloneqq 
\\
\boldsymbol{\mathrm{proc}}\left(a ,b ,c \right)\quad 
\\
\boldsymbol{\mathrm{if}}\quad a =0\quad \boldsymbol{\land}\quad b =0\quad \boldsymbol{\land}\quad c =0\quad \boldsymbol{\mathrm{then}}\quad \mathit{false} \quad \boldsymbol{\mathrm{else}}\quad 
\\
\mathit{dividebyU2} \! \left(\mathit{dividebyA} \! \left(\mathit{dividebyB} \! \left(\mathit{dividebyC} \! \left(\mathit{dividebyD} \! \left(\mathit{dividebyE} \! \left(\mathit{dividebyF} \! \left(a ,b ,c \right)\right)\right)\right)\right)\right)\right)
\\
;\quad 
\\
\boldsymbol{\mathrm{if}}\quad 0<=\mathit{sign1} \quad \boldsymbol{\mathrm{then}}\quad 
\\
\mathit{print} \! \left(a ,b ,
\\
\left[\mathit{sign1} ,\mathit{countU} ,\mathit{countA} ,\mathit{countB} ,\mathit{countC} ,\mathit{countD} ,\mathit{countE} ,
\\
\mathit{countF} \right]\right)\quad 
\\
\boldsymbol{\mathrm{else}}\quad \mathit{print} \! \left(a ,b ,\mathit{False\,-\,does\,not\,factor\,} \right);\quad \mathit{false} \quad \boldsymbol{\textrm{end if}}\quad 
\\
\boldsymbol{\textrm{end if}}\quad 
\\
\boldsymbol{\textrm{end proc}}
\end{dmath}
\mapleinput
{$ \displaystyle \texttt{>\,}  $}

\begin{Maple Normal}
The next two procedures perform multiplication of triples [a,b,c]; mult1 combines a pair of tripes while mult4 takes a string and uses mult1 to combines them pairwise
\end{Maple Normal}
\mapleinput
{$ \displaystyle \texttt{>\,}  $}

\mapleinput
{$ \displaystyle \texttt{>\,} \,\mathit{mult1} \coloneqq \mathrm{proc}(x ,y);\,\mathit{mult2} (x [1],x [2],x [3],y [1],y [2],y [3])\,;\mathrm{end};\, $}

% \mapleresult
\begin{dmath}\label{(18)}
\mathit{mult1} \coloneqq 
\\
\boldsymbol{\mathrm{proc}}\left(x ,y \right)\quad \mathit{mult2} \! \left(x \left[1\right],x \left[2\right],x \left[3\right],y \left[1\right],y \left[2\right],y \left[3\right]\right)\quad 
\\
\boldsymbol{\textrm{end proc}}
\end{dmath}
\mapleinput
{$ \displaystyle \texttt{>\,}  $}

\mapleinput
{$ \displaystyle \texttt{>\,}  $}

\mapleinput
{$ \displaystyle \texttt{>\,} \mathit{mult4} \coloneqq \mathrm{proc}()\mathrm{global}\,L ;\,L \coloneqq [];\,\mathrm{for}\,i \,\mathrm{from}\,1\,\mathrm{to}\,\mathit{nargs} \,\mathrm{do}\,L \coloneqq [\mathit{op} (L),\mathit{args} [i]];\mathrm{od};\,\mathrm{if}\,\mathit{nops} (L)=2\,\mathrm{then}\,\mathit{mult1} (\mathit{op} (1,L),\mathit{op} (2,L))\,\mathrm{else}\,k \coloneqq [\mathit{mult1} (\mathit{op} (1,L),\mathit{op} (2,L))];L \coloneqq \mathit{subsop} (1=\mathit{NULL} ,L);\,L \coloneqq \mathit{subsop} (1=\mathit{NULL} ,L);L \coloneqq [k ,\mathit{op} (L)];\,\mathit{mult4} (\mathit{op} (L));\mathrm{fi};\mathrm{end}; $}

% \mapleresult
\href{http://www.maplesoft.com/support/help/errors/view.aspx?path=Warning,%20(in%20mult4)%20%60i%60%20is%20implicitly%20declared%20local}{Warning, (in mult4) `i` is implicitly declared local}% \mapleresult
\href{http://www.maplesoft.com/support/help/errors/view.aspx?path=Warning,%20(in%20mult4)%20%60k%60%20is%20implicitly%20declared%20local}{Warning, (in mult4) `k` is implicitly declared local}% \mapleresult
\begin{dmath}\label{(19)}
\mathit{mult4} \coloneqq 
\\
\boldsymbol{\mathrm{proc}}\left(\right)\quad \boldsymbol{\mathrm{local}}\quad i ,k ;\quad \boldsymbol{\mathrm{global}}\quad L ;\quad 
\\
L \coloneqq \left[\right];\quad \boldsymbol{\mathrm{for}}\quad i \quad \boldsymbol{\mathrm{to}}\quad \mathit{nargs} \quad \boldsymbol{\mathrm{do}}\quad L \coloneqq \left[\mathit{op} \! \left(L \right),\mathit{args} \left[i \right]\right]\quad \boldsymbol{\textrm{end do}};\quad 
\\
\boldsymbol{\mathrm{if}}\quad \mathit{nops} \! \left(L \right)=2\quad \boldsymbol{\mathrm{then}}\quad \mathit{mult1} \! \left(\mathit{op} \! \left(1,L \right),\mathit{op} \! \left(2,L \right)\right)\quad \boldsymbol{\mathrm{else}}\quad 
\\
k \coloneqq \left[\mathit{mult1} \! \left(\mathit{op} \! \left(1,L \right),\mathit{op} \! \left(2,L \right)\right)\right];\quad L \coloneqq \mathit{subsop} \! \left(1=\mathit{NULL} ,L \right);\quad 
\\
L \coloneqq \mathit{subsop} \! \left(1=\mathit{NULL} ,L \right);\quad L \coloneqq \left[k ,\mathit{op} \! \left(L \right)\right];\quad \mathit{mult4} \! \left(\mathit{op} \! \left(L \right)\right)\quad 
\\
\boldsymbol{\textrm{end if}}\quad 
\\
\boldsymbol{\textrm{end proc}}
\end{dmath}
\mapleinput
{$ \displaystyle \texttt{>\,} \mathit{mult4} (U ,U ,B ,E); $}

% \mapleresult
\begin{dmath}\label{(20)}
1,5,3
\end{dmath}
\mapleinput
{$ \displaystyle \texttt{>\,}  $}

\mapleinput
{$ \displaystyle \texttt{>\,}  $}

\begin{Maple Normal}
Now we use these procedures to try to factorise N = 9263 = 59*157 = 21^3 + 2. Hence r = 21 and we can work in the field Q
{$ ((-2)^{\frac{1}{3}})\mathrm{whose}\mathrm{primes}\mathrm{we}\mathrm{have}\mathrm{studied} $}
\end{Maple Normal}
\mapleinput
{$ \displaystyle \texttt{>\,} N \coloneqq 21^{3}+2; $}

% \mapleresult
\begin{dmath}\label{(21)}
N \coloneqq 9263
\end{dmath}
\mapleinput
{$ \displaystyle \texttt{>\,} \mathit{ifactor} (N); $}

% \mapleresult
\begin{dmath}\label{(22)}
\left(59\right) \left(157\right)
\end{dmath}
\begin{Maple Normal}

\end{Maple Normal}
\begin{Maple Normal}
In the next step we populate a list B with the values of a and b (small) where the factor base for a + 21b contains only the small primes \{2,3,5,7,11,13\}
\end{Maple Normal}
\begin{Maple Normal}
There are 47 pairs (a,b) with a and b between -9 and 9 where both a + 21b and [a,b,0] can be completely factorised using a small factor base. The values of the factors form a 15 column matrix; the first 7 columns are the powers of -1, 2, 3, 5, 7, 11, 13 that factor a + 21b and the last 8 are the powers of -1, U, A, B, C, D, E, F that factorise [a,b,0]. To display the matrix we need to increase the default size to 50x50. 
\end{Maple Normal}
\mapleinput
{$ \displaystyle \texttt{>\,} \mathit{interface} (\mathit{rtablesize} =50) $}

% \mapleresult
\begin{dmath}\label{(23)}
50
\end{dmath}
\mapleinput
{$ \displaystyle \texttt{>\,}  $}

\begin{Maple Normal}
Row	a	b	a+21b\

1	-9	-3	-72\

2	-9	0	-9\

3	-7	-3	-70\

4	-7	1	14\

5	-6	-4	-90\

6	-6	-2	-48\

7	-6	0	-6\

8	-4	-4	-88\

9	-4	4	80\

10	-3	-3	-66\

11	-3	-2	-45\

12	-3	-1	-24\

13	-3	3	60\

14	-2	-3	-65\

15	-2	-2	-44\

16	-2	0	-2\

17	-2	2	40\

18	-1	-1	-22\

19	-1	0	-1\

20	0	-4	-84\

21	0	-3	-63\

22	0	-1	-21\

23	0	2	42\

24	0	3	63
\end{Maple Normal}
\begin{Maple Normal}
25           9              9             198 \

26	0	4	84\

27	1	-1	-20\

28	1	1	22\

29	2	-2	-40\

30	2	2	44\

31	2	3	65\

32	3	-3	-60\

33	3	1	24\

34	3	3	66\

35	4	-4	-80\

36	4	4	88\

37	6	0	6\

38	6	2	48\

39	6	4	90\

40	7	-1	-14\

41	7	3	70\

42	9	0	9\

43	9	3	72\

44	-9	9	180\

45	-8	8	160\

46	6	6	132\

47	8	8	176\

\


\end{Maple Normal}
\mapleinput
{$ \displaystyle \texttt{>\,} R \coloneqq \mathit{Matrix} (47,15,[1,3,2,0,0,0,0,1,2,0,3,2,0,0,0,1,0,2,0,0,0,0,1,2,0,6,0,0,0,0,1,1,0,1,1,0,0,1,1,0,0,0,0,2,0,0,1,0,0,1,0,0,0,0,0,1,1,0,0,1,1,1,2,1,0,0,0,1,1,3,0,0,1,0,0,1,4,3,0,0,0,0,1,1,3,0,2,0,0,0,1,1,1,0,0,0,0,1,1,3,3,0,0,0,0,1,3,0,0,0,1,0,1,1,6,0,0,0,0,0,0,4,0,1,0,0,0,0,0,6,1,0,0,0,0,1,1,1,0,0,1,0,1,2,0,3,0,0,0,0,1,0,2,1,0,0,0,1,1,0,0,0,1,0,0,1,3,1,0,0,0,0,1,1,0,0,2,0,0,0,0,2,1,1,0,0,0,1,1,0,4,0,0,0,0,1,0,0,1,0,0,1,1,0,1,0,0,0,0,1,1,2,0,0,0,1,0,1,1,3,0,0,0,0,0,1,1,0,0,0,0,0,0,0,3,0,0,0,0,0,0,3,0,1,0,0,0,1,0,3,1,0,0,0,0,1,1,0,0,0,1,0,1,1,0,0,0,0,0,0,1,0,0,0,0,0,0,1,0,0,0,0,0,0,0,1,2,1,0,1,0,0,1,0,7,0,0,0,0,0,1,0,2,0,1,0,0,1,1,1,3,0,0,0,0,1,0,1,0,1,0,0,1,0,1,0,0,0,0,0,0,1,1,0,1,0,0,1,0,4,0,0,0,0,0,0,0,2,0,1,0,0,1,1,1,3,0,0,0,0,0,1,2,0,0,1,0,1,2,0,3,0,1,0,0,0,2,1,0,1,0,0,0,0,7,0,0,0,0,0,1,2,0,1,0,0,0,1,0,0,1,0,0,0,0,0,1,0,0,0,1,0,1,1,0,0,0,0,0,0,1,3,0,1,0,0,0,0,0,3,1,0,0,0,0,0,2,0,0,0,1,0,1,1,3,0,0,0,0,0,0,0,0,1,0,0,1,0,0,1,0,0,0,0,1,1,2,1,1,0,0,0,1,1,0,4,0,0,0,0,0,3,1,0,0,0,0,1,1,0,0,0,1,0,0,0,1,1,0,0,1,0,1,2,0,3,0,0,0,0,1,4,0,1,0,0,0,1,0,6,1,0,0,0,0,0,3,0,0,0,1,0,1,1,6,0,0,0,0,0,0,1,1,0,0,0,0,1,1,3,3,0,0,0,0,0,4,1,0,0,0,0,1,1,3,0,2,0,0,0,0,1,2,1,0,0,0,1,1,3,0,0,1,0,0,1,1,0,0,1,0,0,1,0,0,1,1,0,0,1,0,1,0,1,1,0,0,1,1,0,0,0,0,2,0,0,0,2,0,0,0,0,1,2,0,6,0,0,0,0,0,3,2,0,0,0,0,1,2,0,3,2,0,0,0,0,2,2,1,0,0,0,1,2,0,7,0,0,0,0,0,5,0,1,0,0,0,1,0,9,1,0,0,0,0,0,2,1,0,0,1,0,1,2,3,3,0,0,0,0,0,4,0,0,0,1,0,1,1,9,0,0,0,0,0]); $}

% \mapleresult
\begin{dmath}\label{(24)}
R \coloneqq 
\\
\left[\begin{array}{ccccccccccccccc}
1 & 3 & 2 & 0 & 0 & 0 & 0 & 1 & 2 & 0 & 3 & 2 & 0 & 0 & 0 
\\
 1 & 0 & 2 & 0 & 0 & 0 & 0 & 1 & 2 & 0 & 6 & 0 & 0 & 0 & 0 
\\
 1 & 1 & 0 & 1 & 1 & 0 & 0 & 1 & 1 & 0 & 0 & 0 & 0 & 2 & 0 
\\
 0 & 1 & 0 & 0 & 1 & 0 & 0 & 0 & 0 & 0 & 1 & 1 & 0 & 0 & 1 
\\
 1 & 1 & 2 & 1 & 0 & 0 & 0 & 1 & 1 & 3 & 0 & 0 & 1 & 0 & 0 
\\
 1 & 4 & 3 & 0 & 0 & 0 & 0 & 1 & 1 & 3 & 0 & 2 & 0 & 0 & 0 
\\
 1 & 1 & 1 & 0 & 0 & 0 & 0 & 1 & 1 & 3 & 3 & 0 & 0 & 0 & 0 
\\
 1 & 3 & 0 & 0 & 0 & 1 & 0 & 1 & 1 & 6 & 0 & 0 & 0 & 0 & 0 
\\
 0 & 4 & 0 & 1 & 0 & 0 & 0 & 0 & 0 & 6 & 1 & 0 & 0 & 0 & 0 
\\
 1 & 1 & 1 & 0 & 0 & 1 & 0 & 1 & 2 & 0 & 3 & 0 & 0 & 0 & 0 
\\
 1 & 0 & 2 & 1 & 0 & 0 & 0 & 1 & 1 & 0 & 0 & 0 & 1 & 0 & 0 
\\
 1 & 3 & 1 & 0 & 0 & 0 & 0 & 1 & 1 & 0 & 0 & 2 & 0 & 0 & 0 
\\
 0 & 2 & 1 & 1 & 0 & 0 & 0 & 1 & 1 & 0 & 4 & 0 & 0 & 0 & 0 
\\
 1 & 0 & 0 & 1 & 0 & 0 & 1 & 1 & 0 & 1 & 0 & 0 & 0 & 0 & 1 
\\
 1 & 2 & 0 & 0 & 0 & 1 & 0 & 1 & 1 & 3 & 0 & 0 & 0 & 0 & 0 
\\
 1 & 1 & 0 & 0 & 0 & 0 & 0 & 0 & 0 & 3 & 0 & 0 & 0 & 0 & 0 
\\
 0 & 3 & 0 & 1 & 0 & 0 & 0 & 1 & 0 & 3 & 1 & 0 & 0 & 0 & 0 
\\
 1 & 1 & 0 & 0 & 0 & 1 & 0 & 1 & 1 & 0 & 0 & 0 & 0 & 0 & 0 
\\
 1 & 0 & 0 & 0 & 0 & 0 & 0 & 1 & 0 & 0 & 0 & 0 & 0 & 0 & 0 
\\
 1 & 2 & 1 & 0 & 1 & 0 & 0 & 1 & 0 & 7 & 0 & 0 & 0 & 0 & 0 
\\
 1 & 0 & 2 & 0 & 1 & 0 & 0 & 1 & 1 & 1 & 3 & 0 & 0 & 0 & 0 
\\
 1 & 0 & 1 & 0 & 1 & 0 & 0 & 1 & 0 & 1 & 0 & 0 & 0 & 0 & 0 
\\
 0 & 1 & 1 & 0 & 1 & 0 & 0 & 1 & 0 & 4 & 0 & 0 & 0 & 0 & 0 
\\
 0 & 0 & 2 & 0 & 1 & 0 & 0 & 1 & 1 & 1 & 3 & 0 & 0 & 0 & 0 
\\
 0 & 1 & 2 & 0 & 0 & 1 & 0 & 1 & 2 & 0 & 3 & 0 & 1 & 0 & 0 
\\
 0 & 2 & 1 & 0 & 1 & 0 & 0 & 0 & 0 & 7 & 0 & 0 & 0 & 0 & 0 
\\
 1 & 2 & 0 & 1 & 0 & 0 & 0 & 1 & 0 & 0 & 1 & 0 & 0 & 0 & 0 
\\
 0 & 1 & 0 & 0 & 0 & 1 & 0 & 1 & 1 & 0 & 0 & 0 & 0 & 0 & 0 
\\
 1 & 3 & 0 & 1 & 0 & 0 & 0 & 0 & 0 & 3 & 1 & 0 & 0 & 0 & 0 
\\
 0 & 2 & 0 & 0 & 0 & 1 & 0 & 1 & 1 & 3 & 0 & 0 & 0 & 0 & 0 
\\
 0 & 0 & 0 & 1 & 0 & 0 & 1 & 0 & 0 & 1 & 0 & 0 & 0 & 0 & 1 
\\
 1 & 2 & 1 & 1 & 0 & 0 & 0 & 1 & 1 & 0 & 4 & 0 & 0 & 0 & 0 
\\
 0 & 3 & 1 & 0 & 0 & 0 & 0 & 1 & 1 & 0 & 0 & 0 & 1 & 0 & 0 
\\
 0 & 1 & 1 & 0 & 0 & 1 & 0 & 1 & 2 & 0 & 3 & 0 & 0 & 0 & 0 
\\
 1 & 4 & 0 & 1 & 0 & 0 & 0 & 1 & 0 & 6 & 1 & 0 & 0 & 0 & 0 
\\
 0 & 3 & 0 & 0 & 0 & 1 & 0 & 1 & 1 & 6 & 0 & 0 & 0 & 0 & 0 
\\
 0 & 1 & 1 & 0 & 0 & 0 & 0 & 1 & 1 & 3 & 3 & 0 & 0 & 0 & 0 
\\
 0 & 4 & 1 & 0 & 0 & 0 & 0 & 1 & 1 & 3 & 0 & 2 & 0 & 0 & 0 
\\
 0 & 1 & 2 & 1 & 0 & 0 & 0 & 1 & 1 & 3 & 0 & 0 & 1 & 0 & 0 
\\
 1 & 1 & 0 & 0 & 1 & 0 & 0 & 1 & 0 & 0 & 1 & 1 & 0 & 0 & 1 
\\
 0 & 1 & 0 & 1 & 1 & 0 & 0 & 1 & 1 & 0 & 0 & 0 & 0 & 2 & 0 
\\
 0 & 0 & 2 & 0 & 0 & 0 & 0 & 1 & 2 & 0 & 6 & 0 & 0 & 0 & 0 
\\
 0 & 3 & 2 & 0 & 0 & 0 & 0 & 1 & 2 & 0 & 3 & 2 & 0 & 0 & 0 
\\
 0 & 2 & 2 & 1 & 0 & 0 & 0 & 1 & 2 & 0 & 7 & 0 & 0 & 0 & 0 
\\
 0 & 5 & 0 & 1 & 0 & 0 & 0 & 1 & 0 & 9 & 1 & 0 & 0 & 0 & 0 
\\
 0 & 2 & 1 & 0 & 0 & 1 & 0 & 1 & 2 & 3 & 3 & 0 & 0 & 0 & 0 
\\
 0 & 4 & 0 & 0 & 0 & 1 & 0 & 1 & 1 & 9 & 0 & 0 & 0 & 0 & 0 
\end{array}\right]
\end{dmath}
\begin{Maple Normal}

\end{Maple Normal}
\begin{Maple Normal}

\end{Maple Normal}
\begin{Maple Normal}

\end{Maple Normal}
\begin{Maple Normal}
We aim to find sets of rows that are linearly dependent modulo 2; one possibility is rows 23, 37, 41, 45
\end{Maple Normal}
\begin{Maple Normal}

\end{Maple Normal}
\begin{Maple Normal}
These rows give us a factor (59) of N when we calculate the values of u and v that give a congruence u^2 = v^2 mod N and find the gcd of N and u - v
\end{Maple Normal}
\begin{Maple Normal}

\end{Maple Normal}
\end{document}
